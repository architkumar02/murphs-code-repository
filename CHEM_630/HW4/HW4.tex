\documentclass[11pt]{article}

\usepackage{amsmath}
\usepackage{amssymb}
\usepackage{fancyhdr}
\usepackage{isotope}
\usepackage{graphicx}
\graphicspath{{./images/}{./}}
\usepackage{array}
\usepackage{caption}
\usepackage{subcaption}

\oddsidemargin0cm
\topmargin-2cm     %I recommend adding these three lines to increase the 
\textwidth16.5cm   %amount of usable space on the page (and save trees)
\textheight23.5cm  

\newcommand{\question}[2] {\vspace{.25in} \hrule\vspace{0.5em}
\noindent{\bf #1: #2} \vspace{0.5em}
\hrule \vspace{.10in}}
\renewcommand{\part}[1] {\vspace{.10in} {\bf (#1)}}

\newcommand{\myname}{Matthew J. Urffer}
\newcommand{\myemail}{murffer@utk.edu}
\newcommand{\myhwnum}{4}

\pagestyle{fancyplain}
\lhead{\fancyplain{}{\textbf{HW\myhwnum}}}      % Note the different brackets!
\rhead{\fancyplain{}{\myname\\ \myemail}}
\chead{\fancyplain{}{CHEM-630}}

\begin{document}

\medskip                        % Skip a "medium" amount of space
                                % (latex determines what medium is)
                                % Also try: \bigskip, \littleskip

\thispagestyle{plain}
\begin{center}                  % Center the following lines
{\Large CHEM-630 Assignment \myhwnum} \\
\myname \\
\myemail \\
January 24, 2013 \\
\end{center}

\question{2.05}{Radioactivety - Decay Modes}

%%%%%%%%%%%%%%%%%%%%%%%%%%%%%%%%%%%%%%%%%%%%%%%%%%%%%%%
\part{a}{Write the complete equations for the beta decay of \isotope[29]{Al} nad \isotope[140]{La}}

Beta decay is when a neutron in the nuclues decays into a proton, a beta,and an antineutrino:
\begin{align*}
    \text{n} \to \text{p}^{+} + \beta^{-} + \bar{\nu_e}
\end{align*}
\begin{align}
    \isotope[29][13]{Al} & \to \isotope[29][14]{Si}^{+} + \beta^{-} + \bar{\nu_e} \\
    \isotope[140][57]{La} & \to \isotope[140][58]{Ce}^{+} + \beta^{-} + \bar{\nu_e} 
\end{align}

%%%%%%%%%%%%%%%%%%%%%%%%%%%%%%%%%%%%%%%%%%%%%%%%%%%%%%%
\part{b}{Write the complete equations for the positron decay of \isotope[22]{Na} and \isotope[98]{Pd}}

Positron decay (or $\beta^{+}$ decay) is when a proton in the nuclues decays into a neutron, positron, and a neutrino:
\begin{align*}
    \text{p}^{+} \to \text{n} + \beta^{+} + \nu_e
\end{align*}
\begin{align}
    \isotope[22][11]{Na} & \to \isotope[22][10]{Ne}^{-} + \beta^{+} + \nu_e \\
    \isotope[98][46]{Pd} & \to \isotope[98][45]{Rh}^{-} + \beta^{+} + \nu_e
\end{align}


%%%%%%%%%%%%%%%%%%%%%%%%%%%%%%%%%%%%%%%%%%%%%%%%%%%%%%%
\part{c}{Write the complete elquations for the electron capture decay of \isotope[135]{La} and \isotope[212]{Fr}}

Electron capture is when the wave functions of the nuclues and the low level electrons overlap and an electron is pulled into the nuclues, changing a proton in the nuclues to a neutron:
\begin{align*}
    \text{p}^{+} + e^{-} \to \text{n} + \nu_e
\end{align*}
\begin{align}
    \isotope[135][57]{La} +e^{-} &\to \isotope[135][56]{Ba}^{-} + \nu_e \\
    \isotope[212][87]{Fr} +e^{-} &\to \isotope[212][86]{Rn}^{-} + \nu_e
\end{align}


%%%%%%%%%%%%%%%%%%%%%%%%%%%%%%%%%%%%%%%%%%%%%%%%%%%%%%%
\part{d}{Write the complete eqautions for the alpha decay of \isotope[208]{Pb} and \isotope[248]{Fm}}

Alpha decay is when an alpha particle (a helium nuculues) is emitted from a nucleus:
\begin{align*}
    \isotope[A][Z]{X} \to \isotope[A-4][Z-2]{Y} + \isotope[4][2]{\mathnormal{\alpha}}
\end{align*}
\begin{align}
    \isotope[208][82]{Pb} &\to \isotope[204][80]{Hg}^{2e^{-}} + \isotope[4][2]{\mathnormal{\alpha}}^{2+} \\ 
    \isotope[248][100]{Fm} &\to \isotope[244][98]{Cf}^{2e^{-}} + \isotope[4][2]{\mathnormal{\alpha}}^{2+} 
\end{align}


%%%%%%%%%%%%%%%%%%%%%%%%%%%%%%%%%%%%%%%%%%%%%%%%%%%%%%%
\part{e}{Write the complete eqautions for the gamma decay of \isotope[110m]{Ag}}

Gamma decay is when a nucleus in an excited state (meta-stable) emitts a gamma and then goes into a lower energy state.
\begin{align}
    \isotope[110m]{Ag} &\to \isotope[110]{Ag} + \gamma
\end{align}


%%%%%%%%%%%%%%%%%%%%%%%%%%%%%%%%%%%%%%%%%%%%%%%%%%%%%%%
\part{f}{Write the complete eqautions for an internal conversion decay different thant the once given above}

Internal conversion occurs when a nuclues in an excited (meta-stable) state interacts with an electron in the lower orbitals and that electron is ejected from the nuclues. This electron is not of nuclear orgin, but the energy that allowed for its ejection is. An example of internal conversion is provided below:
\begin{align}
    \isotope[99m]{Tc} &\to \isotope[99]{Tc}^{+} + e^{-}
\end{align}


%%%%%%%%%%%%%%%%%%%%%%%%%%%%%%%%%%%%%%%%%%%%%%%%%%%%%%%
\part{g}{Write the complete eqautions for a proton decay different thant the once given above}

Proton decay is the theortical decay of a proton:
\begin{align*}
    \text{p}^{+} &\to e^{+} + \pi^{0} \\
    \text{p}^{+} &\to \mu^{+} + \pi^{0}
\end{align*}
While proton decay has never been observed (see the Advanced Topic section) a hypothetical proton decay was searched for by the Super-K Collaboration:
\begin{align}
    \isotope[1][1]{H} &\to e^{+} +\pi^{0}
\end{align}

%%%%%%%%%%%%%%%%%%%%%%%%%%%%%%%%%%%%%%%%%%%%%%%%%%%%%%%
\question{Advanced Topic}{Proton Decay}
\end{document}

