\documentclass[11pt]{article}

\usepackage{amsmath}
\usepackage{amssymb}
\usepackage{fancyhdr}
\usepackage{isotope}
\usepackage{graphicx}
\graphicspath{{./images/}{./}}
\usepackage{array}
\usepackage{caption}
\usepackage{subcaption}

\oddsidemargin0cm
\topmargin-2cm     %I recommend adding these three lines to increase the 
\textwidth16.5cm   %amount of usable space on the page (and save trees)
\textheight23.5cm  

\newcommand{\question}[2] {\vspace{.25in} \hrule\vspace{0.5em}
\noindent{\bf #1: #2} \vspace{0.5em}
\hrule \vspace{.10in}}
\renewcommand{\part}[1] {\vspace{.10in} {\bf (#1)}}

\newcommand{\myname}{Matthew J. Urffer}
\newcommand{\myemail}{murffer@utk.edu}
\newcommand{\myhwnum}{5}
\newcommand{\iso}{\isotope}

\pagestyle{fancyplain}
\lhead{\fancyplain{}{\textbf{HW\myhwnum}}}      % Note the different brackets!
\rhead{\fancyplain{}{\myname\\ \myemail}}
\chead{\fancyplain{}{CHEM-630}}

\begin{document}

\medskip                        % Skip a "medium" amount of space
                                % (latex determines what medium is)
                                % Also try: \bigskip, \littleskip

\thispagestyle{plain}
\begin{center}                  % Center the following lines
{\Large CHEM-630 Assignment \myhwnum} \\
\myname \\
\myemail \\
January 29, 2013 \\
\end{center}

%%%%%%%%%%%%%%%%%%%%%%%%%%%%%%%%%%%%%%%%%%%%%%%%%%%%%%%
\question{2.07}{Radioactivity - Predict Probable Decay}
\part{a}{Predict the probable decay of \iso[55]{Mn}, \iso[123]{Sn}, \iso[66]{Ga}, \iso[185]{Os}, \iso[242]{Pu}, \iso[55]{Fe}, and \iso[249]{Cm}}

\part{b}{Check the above nuclides in a nuclide table or chart to see how they actually decay\footnote{Decay schemes for this problem where taken from the Nuclear Data Sheets,2003. (http://www.nndc.bnl.gov/chart/chartNuc.jsp)}}
\begin{center}
\begin{tabular}{m{1.5cm} m{5.5cm} m{3cm} m{5.1cm}}
\centering
Isotope & Reason & Predicted Decay Mode & Decay modes \\
\hline
\hline
\iso[55][25]{Mn} & Odd Z, even N with W-A = 0 &  Stable & Stable \\
\iso[123][50]{Sn} & Even Z, odd N with W-A = 4 indicating instability, but Sn has 10 stable isotopes due to begin a magic number &  Stable & $\beta^{-}$: 100.00\% \\
\iso[66][31]{Ga} & Odd, odd nuclide with W-A = -4 indicating instability with a deficiency of protons $\beta^{+}$ &  $\beta^{+}$ & $\beta^{+}$: 100.00\% \\
\iso[185][76]{Os} & Even Z with W-A = -5 indicating instability with a deficiency of protons leading to $\beta^{+}$ &  $\beta^{=}$ & $\beta^{+}$: 100.00\%\\
\iso[242][94]{Pu} & Even Z with W-A = -2, but all elements above Z=82 are unstable, predominately through alpha decay &  $\alpha$ & $\alpha$: 100.00\% , SF: $5.5\times10^{-4}$\% \\
\iso[55][26]{Fe} & Even Z with W-A = 1 &  Stable & $\beta^{+}$: 100.00\% \\
\iso[249][96]{Cm} & Even Z with W-A = 2, but all elements above Z=82 are unstable, predominately through alpha decay &  $\alpha$ & $\beta$: 100.00\% \\
\end{tabular}
\end{center}
The two incorrect guess can be attributed to predicting the instability correctly of the \iso[123]{Sn}, but incorrectly assessing the effects of the magic number in stabilizing.  The \iso[249}{Cm} decay occurs in a competing region of alpha and beta decays.


%%%%%%%%%%%%%%%%%%%%%%%%%%%%%%%%%%%%%%%%%%%%%%%%%%%%%%%
\question{2.09}{Radioactivity - Reaction Energy}

The reaction energies are calculated as the mass difference between the parents and the daughters.
\begin{align*}
	Q = M_{\text{parents}} - M_{\text{daughters}}
\end{align*}
The mass excess of the nuclides, $\Delta$, (differences between actual mass and atomic number) was used instead of the actual nuclide mass. 
The nuclide mass, if desired could be recovered by adding on the mass excess.
It is assumed that the mass of the neutrino is negligible.

\part{a}{Calculate the reaction energy involved in the beta decay of \iso[110]{Ag}}
\begin{align}
	\iso[110][47]{Ag} &\to \iso[110][48]{Cd}^{+} +\beta +\bar{\nu} + Q \notag\\
	Q &= M_{\iso[110]{Ag}} - \left (M_{\iso[110]{Cd}} - M_{e^{-}} + \right ) - M_{e^{-}} - M_{\bar{\nu}} \notag\\
	  &= \Delta_{\iso[110]{Ag}} -\Delta_{\iso[110]{Cd}} \notag \\
	  &= -87.4574 \text{MeV} - \left ( -90.3503 \text{MeV} \right) \notag \\
	  &= 2.8929\text{MeV}
\end{align}

\part{b}{Calculate the reaction energy involved in the positron decay of \iso[98]{Rh}}
\begin{align}
	\iso[98][45]{Rh} &\to \iso[98][44]{Ru}^{-} + \beta^{+} + \nu + Q \notag \\
	Q &= M_{\iso[98]{Rh}} - \left ( M_{\iso[98]{Ru}} + M_{e^{-}} \right ) - M_{e^{-}} -M_{\nu} \notag \\
	  &= \Delta_{\iso[98]{Rh}} - \Delta_{\iso[98]{Ru}} \notag \\
	  &= -83.1751\text{MeV} - \left ( -88.2248\text{MeV} \right ) \notag \\
	  &= 5.057 \text{MeV}
\end{align}

\part{c}{Calculate the reaction energy involved in the electron-capture decay of \iso[111]{Sn}}
\begin{align}
	\iso[110][50]{Sn} + e^{-} &\to \iso[110][49]{In} + \bar{\nu} \notag \\
	Q &= M_{\iso[110]{Sn}} - \left (M_{\iso[110]{In}} + M_{e^{-}} \right ) - M_{e^{-}} -M_{\nu} \notag \\
	  &= \Delta_{\iso[110]{Sn}} - \Delta_{\iso[110]{In}} \notag \\
	  &= -85.9414\text{MeV} - \left (-88.3832\text{MeV} \right ) \notag \\
	  &= 2.451\text{MeV}
\end{align}

\part{d}{Calculate the reaction energy involved in the alpha decay of \iso[148]{Gd}}
\begin{align}
	\iso[148][64]{Gd} &\to \iso[144][62]{Sm}^{2+} +\alpha^{2-} \notag \\
	Q &= M_{\iso[148]{Gd}} - \left (M_{\iso[114]{Sm}}  + 2m_{e^{-}} \right ) - \left ( M_{\iso[4]{He}} - 2m_{e^{-}} \right ) \notag \\
	  &= \Delta_{\iso[148]{Gd}} - \Delta_{\iso[114]{Sm}} - \Delta_{\iso[4]{He}} \notag \\
	  &= -76.2692\text{MeV} -\left (-81.9657\text{MeV} + 2.4249\text{MeV} \right ) \notag \\
	  &= 3.2712\text{MeV}
\end{align}

\part{e}{Calculate the reaction energy involved in the gamma decay of \iso[107m]{Ag}}

The reaction energy involved in the decay of a meta-stable isotope (such as \iso[107m]{Ag}) cannot be calculated from a mass difference as masses are only listed for stable states. 
However,the Q-value for this reaction is 931 keV. 

%%%%%%%%%%%%%%%%%%%%%%%%%%%%%%%%%%%%%%%%%%%%%%%%%%%%%%%
\question{Advanced Topic}{Usage of nuclear isomers in Nuclear Batteries}
Generally nuclear batteries are thought as radioisotope thermoelectric generator (RTG) in which the radioactive decay of an isotope (mostly \iso[238]{Pu} and \iso[90]{Sr}) provides the energy for the device.
The kinetic energy of the daughter reaction products is transferred into heat as the reaction products slow down, which is then converted into energy by thermocouples.
These devices are limited in their use to the large amount of activity needed to power them and the inefficiency of the thermocouples (usually around 10 \%).
Metastable nuclear isomers are proposed as a possible replacement technology in which the isomers are excited into the metastable state and a controlled de-excitation is used to extract power from the battery.  
As metastable isomers decay via gamma emission the thermal energy to electrical energy is not practical due to the interaction mechanisms of photons.
Instead, it is proposed to use these generators in conjunction with a p-n semiconductor so the ionizing radiation creates electron-hole pairs from which current could be drawn.
The majority of the current research has then focused on the ability to control the excitation of the metastable isotopes, as evidenced in the attached papers.

Sources:
\begin{itemize}
\item Wiki - \verb+http://en.wikipedia.org/wiki/Radioisotope_thermoelectric_generator+
\item Wiki - \verb+http://en.wikipedia.org/wiki/Nuclear_isomer+
\end{itemize}

\end{document}

