
%%%%%%%%%%%%%%%%%%%%%%%%%%%%%%%%%%%%%%%%%%%%%%%%%%%%%%%%%%%%%%%%%%%%%%%%%%%%%%%%
%                                                                              %
%                                   PREAMBLE                                   %
%                                                                              %
%%%%%%%%%%%%%%%%%%%%%%%%%%%%%%%%%%%%%%%%%%%%%%%%%%%%%%%%%%%%%%%%%%%%%%%%%%%%%%%%
\documentclass[conference]{IEEEtran}
\usepackage{listings,graphicx,amsmath}
\usepackage{microtype,todonotes}
\usepackage[margin=1in]{geometry}
\usepackage{hyperref}


%% GRAPHICS RELATED
\usepackage{graphicx}
\usepackage{listings,amsmath}
\usepackage{microtype,todonotes}
\graphicspath{{./images/}}
\DeclareGraphicsExtensions{.pdf, .jpeg, .png}

%% CAPTION SETUP
\usepackage{caption}
\usepackage{subcaption}
\captionsetup{belowskip=12pt,aboveskip=4pt}

\usepackage{listings}
\lstset{ %
language=Matlab,                % choose the language of the code
basicstyle=\footnotesize,       % the size of the fonts that are used for the code
%numbers=left,                   % where to put the line-numbers
%numberstyle=\footnotesize,      % the size of the fonts that are used for the line-numbers
%stepnumber=1,                   % the step between two line-numbers. If it is 1 each line will be numbered
%numbersep=5pt,                  % how far the line-numbers are from the code
%backgroundcolor=\color{},  % choose the background color. You must add \usepackage{color}
showspaces=false,               % show spaces adding particular underscores
showstringspaces=false,         % underline spaces within strings
showtabs=false,                 % show tabs within strings adding particular underscores
frame=single,           % adds a frame around the code
tabsize=2,          % sets default tabsize to 2 spaces
captionpos=b,           % sets the caption-position to bottom
breaklines=true,        % sets automatic line breaking
breakatwhitespace=false,    % sets if automatic breaks should only happen at whitespace
escapeinside={\%*}{*)}          % if you want to add a comment within your code
}

%%%%%%%%%%%%%%%%%%%%%%%%%%%%%%%%%%%%%%%%%%%%%%%%%%%%%%%%%%%%%%%%%%%%%%%%%%%%%%%%
%                                                                              %
%                                 Start of Document                            %
%                                                                              %
%%%%%%%%%%%%%%%%%%%%%%%%%%%%%%%%%%%%%%%%%%%%%%%%%%%%%%%%%%%%%%%%%%%%%%%%%%%%%%%%
\begin{document}
\title{Invisible Edge\\
CS 526 Challenge Problem 3, Group 11}

% author names and affiliations
% use a multiple column layout for up to three different
% affiliations
\author{\IEEEauthorblockN{John Martin}
\and
\IEEEauthorblockN{Matthew Urffer}
\and
\IEEEauthorblockN{Lipeng Wan}
}


\maketitle
\IEEEpeerreviewmaketitle
\begin{abstract}
Communication companies have a large amount of information on their consumers.
In order to accurately bill their costumers they need to collect the amount to calls they made (including the date and time), the length of the call, the texts the costumers sent and received.
This data can then be mined in order to determine patterns between their costumer's in order to improve their service, investigate alternative products, or find new customers to pursue.
Data provided by Link Analytics was mined for the invisible edge between costumers; i.e. at what probability can it be determined if two customers might communicate.
This was completed using an LSA based approach, a Neural Network based approach, and a spectral graph.
None of these methods were successful, so it is concluded that more advanced graph theory concepts are needed, but the run time
of these algorithms are generally prohibitive on such a large problem size.
\end{abstract}

% Sections (Other Documents)
\section{Introduction}

A supervised machine learning problem is one which a learning algorithm is presented a set of training data and attempts to find an unknown function which maps the training values to the correct answer.
Typically the training set, denoted $S$, is a set of the form $\left \{ (\vec{x}_1,y_1), (\vec{x}_2,y_2), \dots, (\vec{x}_n,y_n) \right \}$ where $\vec{x}_i$ is vector of some features of the problem.
Examples of problem features include discrete or real valued items such as height, weight, age, zip code, grade point average, starting salary, and telephone number (as just a few)  which might make up the features of a person.
The $y_i$ are the class of the training feature $\vec{x}_i$ belongs to; these might be University of Tennessee students or Carnegie Mellon students.
In this examples students with a zip code of 15213 are likely to be Tartans, while students with a zip code of 37916 are like to be Volunteers.
The challenge arises from examples have overlapping features; for example this author as a former Tartan and current Vol would be difficult to classify by zip code.
The learning algorithms job is then to find a hypothesis $h$ that correctly classifies a student as a Volunteer of Tartan based on the features provided.
This learning process can then be defined as finding the hypothesis that has the least error (incorrect classifications) on the training data set while extending to examples outside of the training space.

\subsection{Support Vector Machines}
Support Vector Machines (SVM) are a supervised learning technique in which hyperplanes are constructed in a high dimensional space to which the features are mapped.
SVMs find the hyperplanes that are the farthest away from all of mapped features in order to provide excellent training performance while still maintaining the ability to generalize to new instances; i.e. SVMs are maximal margin classifiers.
For a binary classification the decision function of the SVM is the dot product of the weight vector and the training example in the feature space added to a bias vector as shown in Equation \ref{eq:BCSVM}.
\begin{equation}
\label{eq:BCSVM}
f \left ( \vec{x} \right ) = \left \langle \vec{w} \phi(\vec{x}) \right \rangle + \vec{b}
\end{equation}
where $\phi(\vec{x})$ is a mapping to the higher dimensional space.
The SVM is then learning the optimal values of the weight vector $\vec{w}$ and the basis $\vec{b}$.

The radial basis function (Equation \ref{eq:RBF}) is a common kernel function used to map the input vector $\vec{x}$ into a higher dimension.
\begin{equation}
\label{eq:RBF}
k \left ( \vec{x}_i , \vec{x}_j \right ) = exp \left ( - \frac{\left \| \vec{x}_i - \vec{x}_j \right \|}{2\sigma^2} \right ) 
\end{equation}
The maximal margin is ensured by minimizing:
\begin{equation}
\label{eq:Min}
g(\vec{w},\eta) = \frac{1}{2} \left \| \vec{w} \right \| + C \sum_{i=1}^N \zeta_i
\end{equation}
subject to:
\begin{equation}
\label{eq:Constraint}
y_i( \left \langle \vec{w},\phi(\vec{x}) \right \rangle + b ) \ge 1-\zeta_i, ~~~\zeta_i \ge 0
\end{equation}
where $\zeta_i$ is the $i$th slack variable and C is the regularization parameter \cite{li_adaboost_2008}.
This problem can be translated  in to the Wolfe dual form, which can be solved with quadratic programing \cite{li_adaboost_2008}.

\subsection{Boosting}
Unbalanced data sets (data sets in which a majority of the values come from one class, see Figure \ref{fig:ClassDist}) are difficult for classification schemes to learn because the minority class is not well represented and tends to be thought as noise for the classifier.
Often classifiers are trained from unbalanced data sets by artificially balancing the data set by sampling techniques; i.e. up-sampling (sampling more from the minority class) and down-sampling (sampling less from the majority class).
Boosting is an ensemble learning method in which a set of weights is maintained over the training samples and adaptively adjusted after each training iteration according to the ones that are misclassified \cite{li_adaboost_2008}.
Given an individual classifier $h$, an ensemble of classifiers can be constructed of a set of individual classifiers, $H={h_1, h_2,\dot, h_n}$.
By maintaining a weight distribution over all of the training examples, these weights could be updated to emphasize the training examples that are misclassified incorrectly.  These incorrectly classified examples could then be learned in a refinement of the classifier or by training adding a new classifier to the ensemble with the new weights.
Performance of the ensemble is enhanced as long as the individual classifiers are weak and have uncorrelated errors as when any single classifier is incorrect the other classifiers in the ensemble might correctly classify the example.
\begin{figure*}[ht!]
	\centering
	\begin{subfigure}[b]{0.3\textwidth}
		\centering
		\includegraphics[width=\textwidth]{Liver_ClassDist}
        \caption{Liver}
	\end{subfigure}%
	~
	\begin{subfigure}[b]{0.3\textwidth}
		\centering
		\includegraphics[width=\textwidth]{Glass_ClassDist}
        \caption{Glass}
	\end{subfigure}	
    ~
	\begin{subfigure}[b]{0.3\textwidth}
		\centering
		\includegraphics[width=\textwidth]{Vowel_ClassDist}
        \caption{Vowel}
	\end{subfigure}%
	\caption{Distribution of Class Data}
	\label{fig:ClassDist}
\end{figure*}

%%%%%%%%%%%%%%%%%%%%%%%%%%%%%%%%%%%%%
\section{Methods}
\label{sec:Methods}

For convince a subversion repository was created to manage the developed code base, and all source code is available by anonymous checkout from \verb+http://www.murphs-code-repository.googlecode.com/svn/trunk/layeredPolymerTracking+. Revision 360 was the code base used to generate the results shown in \ref{sec:Results}.

\subsubsection{Detector Geometry}
The geometry was setup such that it is possible to define multiple layers of detectors, as shown in Figure \ref{fig:LayerDetectorGeo}.
This was done by creating a 
\begin{figure} 
    \includegraphics[width=\figurewidth]{10LayerGamma}
	\caption{10 Layer Detector with a simulated gamma event}
    \label{fig:LayerDetectorGeo}
\end{figure}
\subsubsection{Physics Lists}

\section{Results}
\label{sec:Results}

\subsection{Parmater Search}

The parameter search for the optimal C and $\sigma$ parameters is shown in the contour plots of Figures \ref{fig:ParamLiver}, \ref{fig:ParamGlass} and \ref{fig:ParamGlass}.
The optimal classifier parameters are shown for the coarse parameter search in Table \ref{tab:CoarseParamValues} and for the fine parameter search in Table \ref{tab:FineParamValues}.
\todo{Make some quantifications and discuss the data}
\begin{table}[h!]
\caption{Coarse Optimal Classifier Parameters}
\label{tab:CoarseParamValues}
\begin{tabular}{c c c c c c c c}
\hline
Data Set & $C_{min}$ & $C_{max}$ & $\sigma_{min}$ & $\sigma_{max}$ & $C$ & $\sigma$ & $\epsilon$ \\ 
\hline
Glass & -5.00 & 5.00 & -5.00 & 5.00 & 5.00 & 0.56 & 72.78 \\ 
Liver & -5.00 & 5.00 & -5.00 & 5.00 & 3.89 & -2.78 & 74.33 \\ 
Vowel & -5.00 & 5.00 & -5.00 & 5.00 & 5.00 & 2.78 & 99.24 \\ 
\hline
\end{tabular}
\end{table}
\begin{table}[ht]
\caption{Fine Optimal Classifier Parameters}
\label{tab:FineParamValues}
\begin{tabular}{c c c c c c c c}
Data Set & $C_{min}$ & $C_{max}$ & $\sigma_{min}$ & $\sigma_{max}$ & $C$ & $\sigma$ & $\epsilon$ \\ 
\hline
Glass & 2.50 & 7.50 & 0.28 & 0.83 & 6.71 & 0.31 & 75.00 \\ 
Liver & 1.94 & 5.83 & -4.17 & -1.39 & 3.79 & -1.68 & 75.67 \\ 
Vowel & 2.50 & 7.50 & 1.39 & 4.17 & 7.50 & 1.97 & 99.43 \\ 
\hline
\end{tabular}
\end{table}
\begin{figure*}[ht!]
	\centering
	\begin{subfigure}[b]{0.45\textwidth}
		\centering
		\includegraphics[width=\textwidth]{Liver_coarseSearch}
        \caption{Coarse Search}
	\end{subfigure}%
	~
	\begin{subfigure}[b]{0.45\textwidth}
		\centering
		\includegraphics[width=\textwidth]{Liver_fineSearch}
        \caption{Fine Search}
	\end{subfigure}	
	\caption{Parameter search for Liver Disorder}
	\label{fig:ParamLiver}

	\begin{subfigure}[b]{0.45\textwidth}
		\centering
		\includegraphics[width=\textwidth]{Glass_coarseSearch}
        \caption{Coarse Search}
	\end{subfigure}%
	~
	\begin{subfigure}[b]{0.45\textwidth}
		\centering
		\includegraphics[width=\textwidth]{Glass_fineSearch}
        \caption{Fine Search}
	\end{subfigure}	
	\caption{Parameter search for Glass Disorder}
	\label{fig:ParamGlass}

	\begin{subfigure}[b]{0.45\textwidth}
		\centering
		\includegraphics[width=\textwidth]{Vowel_coarseSearch}
        \caption{Coarse Search}
	\end{subfigure}%
	~
	\begin{subfigure}[b]{0.45\textwidth}
		\centering
		\includegraphics[width=\textwidth]{Vowel_fineSearch}
        \caption{Fine Search}
	\end{subfigure}	
	\caption{Parameter search for Vowel Disorder}
	\label{fig:ParamVowel}
\end{figure*}

\subsection{AdaBoostM1}

\section{Conclusions}

It is concluded that due to the lack of negative training examples (edges that should not be there) any classification based system will preform with an accuracy of 50\%.

\section{Acknowledgments}

The work was distributed as follows. John completed the analysis of the data with the LSI based similarity measure.
Matthew completed the distributions of the data as well as the neural network classification system. He also implemented the probability based approach.
Lipeng completed the work on the spectral graph density.
Many thanks to Link Analytics for providing this unique data set.


\end{document}
