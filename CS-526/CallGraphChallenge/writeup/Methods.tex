\section{Methods}
\label{sec:Methods}

The call graph was provided as a labeled edge list, and was modeled as a weighted, undirected, multigraph 
The graph was organzied as each of the nodes being the unique identifier to a cutomer, with edges between the customers representing:
\begin{itemize}
	\item \texttt{days} - the number of distinict days that the two nodes communicated during the month,
	\item \texttt{calls} - the number of distinct calls that the the nodes made during the month,
	\item \texttt{secs} - the cummulative sum of all calls that the modes made during the month and,
	\item \texttt{texts} - the number of distinct texts that the two nodes made during the month.
\end{itemize}
\missingfigure{A simple call graph}

\subsection{LSA Based Similarity Measure}
% Copied from John's email
Based on some work we did a few years ago looking at personality profile
matching using an LSA based system, I wondered if the premise might hold that
a node (person) could be effectively "described" by the set of other people
they were connected to.  This description could then be measured for
similarity to other node descriptions and hopefully we would find that
"similar" nodes were connected. In the past work we did there was some promise
in the initial results for group analysis, but it was not pursued very far and
there were other data items that were being considered as well (and it had
nothing to do with telephone usage patterns).

To test the theory, I initially constructed a sparse matrix for each of the
graph files for both the Moria and Standelf.  For each connected node, the
connection weight was determined by taking each of the 4 attribute values,
dividing by thier respective standard deviation, and summing the results into
a single weight for each connection. I also reran the experiment using a
simpler weighting scheme giving a 1 if any call was made and another 1 if any
text was made.

This sparse matrix in any case was extremely sparse, being around 0.00051\%
nonzero for Moria and 0.00059\% for Standelf.  We usually deal with matrices
that have a nonzero rate of 0.001\% to 0.01\% for text mining applications.  I
computed a truncated SVD for these sparse matrices, forming an LSA space at
approximately 250 dimensions for each.  Then using these 250 dimenstion spaces
I looked at comparing the first 1,000 nodes of each graph to all the other
nodes in the graph computing their vector cosine similiarity and noting if the
nodes were connected or not.  This takes a while to run so there has not been
much opportunity to tweak any of the parameters, therefore I cannot conclude
with any certainty, but the initial results do not look promising.  The first
sets that I  processed did not show any clear indication of connectivity
determined by the similarity measure.

\subsection{Artifical Neural Network Classification}
The next approach implemented was an artifical neural network classification system.
Using the \verb+PyBrain+ tool module a feed forward neural network was constructed with 8 inputs represting the edge between node \texttt{u} and \texttt{v}:
\begin{itemize}
	\item the degree of node \texttt{u},
	\item the closeness of node \texttt{u},
	\item \texttt{days} of the edge,
	\item \texttt{calls} of the edge,
	\item \texttt{secs} of the edge,
	\item \texttt{texts} of the edge,
	\item the closeness of node \texttt{v}, and,
	\item the degree of node \texttt{u}.
\end{itemize}
The closeness of node was calucated using the \verb+networkx+ module with \verb+closeness_centrality+, defined to be 1 over the average distance to all other nodes. The distances where not weighted, and all distances where normalized by the graph size.
The degree was calculated with \verb+degree+ as the the sum of the edge weights of adjacent nodes for a particular node (completed for all nodes).
The other elements of the training vector were simply filled with the values from the edge.

Training data for the neural network was then the set of all example nodes and edges presented in the target class of 1 (the edge exists). This was ultimately a flaw in the design of the neural network (or any classification based system) as discused in Section \ref{sec:Results}.
