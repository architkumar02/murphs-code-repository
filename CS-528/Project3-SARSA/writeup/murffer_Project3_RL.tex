\documentclass[conference]{IEEEtran}

% *** MISC UTILITY PACKAGES ***
%


% *** CITATION PACKAGES ***
%
\usepackage{cite}

% *** GRAPHICS RELATED PACKAGES ***
%
\usepackage[pdftex]{graphicx}
% declare the path(s) where your graphic files are
\graphicspath{images}
\DeclareGraphicsExtensions{.pdf,.jpeg,.png}

% *** MATH PACKAGES ***
%
\usepackage[cmex10]{amsmath}

% *** SPECIALIZED LIST PACKAGES ***
%
\usepackage{algorithmic}

% *** ALIGNMENT PACKAGES ***
%
\usepackage{array}


% *** SUBFIGURE PACKAGES ***
\usepackage[cpation=false,font=footnotesize]{subfig}

% *** FLOAT PACKAGES ***
%

% *** PDF, URL AND HYPERLINK PACKAGES ***
%
\usepackage{url}


% correct bad hyphenation here
\hyphenation{op-tical net-works semi-conduc-tor}


\begin{document}
%
% paper title
% can use linebreaks \\ within to get better formatting as desired
\title{Wall following and obstacle avoidance with SARSA $\lambda$ Reinforcement Learning}


% author names and affiliations
% use a multiple column layout for up to three different
% affiliations
\author{\IEEEauthorblockN{Matthew J. Urffer}
\IEEEauthorblockA{Department of Nuclear Engineering \\
University of Tennessee \\
Knoxvilee, Tennessee, 37916 \\
Email: matthew.urffer@gmail.com
}}



% make the title area
\maketitle


\begin{abstract}
\boldmath
\begin{itemize}
	\item Reinforcment learning has many applictions in learning control strategies.
	\item Something about trying to find the policy which has the optimal award
	\item SARSA lambda was implemented for a wall following and obstical avoidence using the Player-Stage simulation enviroment
	\item A discritized space was used
\end{itemize}
\end{abstract}
\IEEEpeerreviewmaketitle



\section{Introduction}

Reinforcement learning is a popular learning strategy for which direct supervision of the agent is not possible.  Rather than learning a particular solution to the problem, Reinformencet learning attempts to find a policy for solving the problem.  

\subsection{Reinforcment Learning}
The \textbf{agent} in reinforcemnt must go out in the enviorment and complete the following:
\begin{itemize}
	\item find out the \textbf{state} of the environment,
	\item take \textbf{actions} to modify the enviroment,
	\item determine if the \textbf{goal} has been acheived.
\end{itemize}
The agent then interacts within the enviroment in order to achieve a goal.  In reinforcemnt learning this is broken into four sections:
\begin{itemize}
	\item a \textbf{policy} which maps between a state and and action that the agent can preform,
	\item a \textbf{reward function} which defines the goal of the agent,
	\item a \textbf{value function} which determines how profitable it is to be in a given state (in relationship to achieving the award)
	\item and a \textbf{enviroment} which determines the states and actions that can be completed.
\end{itemize}

\textbf{FIGURE OF AGENT AND MODEL INTERACTION}




\section{Things to try}
\begin{itemize}
	\item What happens if we expand the discreitization
	\item Can we make two policies, one for avoidance and one for wall following
\end{itemize}

\hfill mds
 
\hfill January 11, 2007


% An example of a floating figure using the graphicx package.
% Note that \label must occur AFTER (or within) \caption.
% For figures, \caption should occur after the \includegraphics.
% Note that IEEEtran v1.7 and later has special internal code that
% is designed to preserve the operation of \label within \caption
% even when the captionsoff option is in effect. However, because
% of issues like this, it may be the safest practice to put all your
% \label just after \caption rather than within \caption{}.
%
% Reminder: the "draftcls" or "draftclsnofoot", not "draft", class
% option should be used if it is desired that the figures are to be
% displayed while in draft mode.
%
%\begin{figure}[!t]
%\centering
%\includegraphics[width=2.5in]{myfigure}
% where an .eps filename suffix will be assumed under latex, 
% and a .pdf suffix will be assumed for pdflatex; or what has been declared
% via \DeclareGraphicsExtensions.
%\caption{Simulation Results}
%\label{fig_sim}
%\end{figure}

% Note that IEEE typically puts floats only at the top, even when this
% results in a large percentage of a column being occupied by floats.


% An example of a double column floating figure using two subfigures.
% (The subfig.sty package must be loaded for this to work.)
% The subfigure \label commands are set within each subfloat command, the
% \label for the overall figure must come after \caption.
% \hfil must be used as a separator to get equal spacing.
% The subfigure.sty package works much the same way, except \subfigure is
% used instead of \subfloat.
%
%\begin{figure*}[!t]
%\centerline{\subfloat[Case I]\includegraphics[width=2.5in]{subfigcase1}%
%\label{fig_first_case}}
%\hfil
%\subfloat[Case II]{\includegraphics[width=2.5in]{subfigcase2}%
%\label{fig_second_case}}}
%\caption{Simulation results}
%\label{fig_sim}
%\end{figure*}
%
% Note that often IEEE papers with subfigures do not employ subfigure
% captions (using the optional argument to \subfloat), but instead will
% reference/describe all of them (a), (b), etc., within the main caption.


% An example of a floating table. Note that, for IEEE style tables, the 
% \caption command should come BEFORE the table. Table text will default to
% \footnotesize as IEEE normally uses this smaller font for tables.
% The \label must come after \caption as always.
%
%\begin{table}[!t]
%% increase table row spacing, adjust to taste
%\renewcommand{\arraystretch}{1.3}
% if using array.sty, it might be a good idea to tweak the value of
% \extrarowheight as needed to properly center the text within the cells
%\caption{An Example of a Table}
%\label{table_example}
%\centering
%% Some packages, such as MDW tools, offer better commands for making tables
%% than the plain LaTeX2e tabular which is used here.
%\begin{tabular}{|c||c|}
%\hline
%One & Two\\
%\hline
%Three & Four\\
%\hline
%\end{tabular}
%\end{table}


% Note that IEEE does not put floats in the very first column - or typically
% anywhere on the first page for that matter. Also, in-text middle ("here")
% positioning is not used. Most IEEE journals/conferences use top floats
% exclusively. Note that, LaTeX2e, unlike IEEE journals/conferences, places
% footnotes above bottom floats. This can be corrected via the \fnbelowfloat
% command of the stfloats package.



\section{Conclusion}
The conclusion goes here.




% conference papers do not normally have an appendix


% use section* for acknowledgement
\section*{Acknowledgment}


The authors would like to thank...





% trigger a \newpage just before the given reference
% number - used to balance the columns on the last page
% adjust value as needed - may need to be readjusted if
% the document is modified later
%\IEEEtriggeratref{8}
% The "triggered" command can be changed if desired:
%\IEEEtriggercmd{\enlargethispage{-5in}}

% references section

% can use a bibliography generated by BibTeX as a .bbl file
% BibTeX documentation can be easily obtained at:
% http://www.ctan.org/tex-archive/biblio/bibtex/contrib/doc/
% The IEEEtran BibTeX style support page is at:
% http://www.michaelshell.org/tex/ieeetran/bibtex/
%\bibliographystyle{IEEEtran}
% argument is your BibTeX string definitions and bibliography database(s)
%\bibliography{IEEEabrv,../bib/paper}
%
% <OR> manually copy in the resultant .bbl file
% set second argument of \begin to the number of references
% (used to reserve space for the reference number labels box)
\begin{thebibliography}{1}

\bibitem{IEEEhowto:kopka}
H.~Kopka and P.~W. Daly, \emph{A Guide to \LaTeX}, 3rd~ed.\hskip 1em plus
  0.5em minus 0.4em\relax Harlow, England: Addison-Wesley, 1999.

\end{thebibliography}




% that's all folks
\end{document}


