%%%%%%%%%%%%%%%%%%%%%%%%%%%%%%%%%%%%%
\section{Introduction to GEANT4}
\label{sec:G4Intro}

GEANT4 (GEomentry ANd Tracking) is a free, open source, Monte Carlo based physics simulation toolkit developed and maintained at CERN widely used in the physics community \cite{agostinelli_geant4simulation_2003,geant4_collaboration_geant4_2011,allison_geant4_2006}.
It is based off of the exsisting FORTRAN based GEANT3, but updated to an object-oriented C++ environment based on an initiative started in 1993.
The initiative grew to become an international collaboration of researchers participating in a range of high-energy physics experiments in Europe, Japan, Canada and the United States. 
As GEANT4 is a toolkit primarily developed for high energy physics, particles are designated according the \verb+PDG+ (Particle Data Group) encoding.
In addition, the physics processes are referenced according to the standard model.
In the standard model particles are divided into two families, bosons (the force carriers such as photons) and fermions (matter).
The fermions consist of both hadrons and leptons.
Hadrons are particles composed of quarks which are divided into two classes: baryons (three quarks) and mesons (two quarks).h
Typical baryons include the neutron and the proton, while an example of a meson is the pion.
An example of a lepton is the electron.
\subsection{Organization of the GEANT4 Toolkit}

The GEANT4 toolkit is divided into eight class categories:
\begin{itemize}
    \item Run and Event - generation of events and secondary particles.
    \item Tracking and Track - transport of a particle by analyzing the factors limiting the step size and by applying the relevant physics models.
    \item Geometry and Magnetic Field - the geometrical definition of a detector (including the computation of the distances to solids) as wells as the management of magnetic fields.
    \item Particle Definition and Matter - definition of particles and matter.
    \item Hits and Digitization - the creation of hits and their use for digitization in order to model a detector's readout response.
    \item Visualization - the visualization of a simulation including the solid geometry, trajectories and hits.
    \item Interface - the interactions between the toolkit and graphical user interfaces and well as external software.
\end{itemize}

There are then three classes which must be implemented by the user in order use the toolkit. These classes are:
\begin{itemize}
    \item \verb+G4VUserDetectorConstruction+ which defines the geometry of the simulation,
    \item \verb+G4VUserPhysicsList+ which defines the physics of the simulation, and
    \item \verb+G4VUserPriamryGeneratorAction+ which defines the generation of primary events.
\end{itemize}
Five additional classes are available for further control over the simulation:
\todo{Expand this section}
\begin{itemize}
    \item \verb+G4UserRunAction+ which allows for user actions
\end{itemize}
\subsection{GEANT4 Tracking and Secondaries}

A GEANT4 simulation starts with a run which contains a set number of events.
In GEANT4 the \verb+Run+ is the large unit of simulation (represented with a \verb+G4Run+ object), which consists of a sequence of events.
An event is particular process of interest to the user, such as shooting a single particle at a detector. 
Typical usage might be to have a run firing 1,000 neutrons at a detector, where each neutron is a single event.
Each particle transported in GEANT4 is assigned a unique track ID and a parent ID.
The particle that initiates the event is given a parent ID of 0 and a track ID of 1.
If the parent particle has a collision, and produces a secondary particle, this secondary particle is then given a parent ID of 1 (corresponding to the first secondary) and a track ID of 2.
Secondaries are tracked in GEANT4 utilizing a stack in which the most recent secondary (and its cascade) is tracked first.

Listing \ref{lst:TrackingExample} provides an example from the verbose output of GEANT4 of the tracking.
The initial particle in the event is the neutron because it has a parent ID of 0.
The alpha and triton are the secondaries produced by this collision. 
The alpha is assigned a parent ID of 1 (corresponding to the first generation) with a track ID of 3.
The triton is also assigned a parent ID of 1, but with a track ID of 2.
\begin{lstlisting}[language=,basicstyle=\tiny,caption={Tracking Example},label=lst:TrackingExample]
*********************************************************************************************************
* G4Track Information:   Particle = neutron,   Track ID = 1,   Parent ID = 0
*********************************************************************************************************

Step#    X(mm)    Y(mm)    Z(mm) KinE(MeV)  dE(MeV) StepLeng TrackLeng  NextVolume ProcName
    0        0        0    -6.59   2.5e-08        0        0         0    Absorber initStep
    1        0        0    -3.64         0        0     2.95      2.95    Absorber NeutronInelastic
    :----- List of 2ndaries - #SpawnInStep=  2(Rest= 0,Along= 0,Post= 2), #SpawnTotal=  2 ---------------
    :         0         0     -3.64      2.73             triton
    :         0         0     -3.64      2.05              alpha
    :----------------------------------------------------------------- EndOf2ndaries Info ---------------

*********************************************************************************************************
* G4Track Information:   Particle = alpha,   Track ID = 3,   Parent ID = 1
*********************************************************************************************************

Step#    X(mm)    Y(mm)    Z(mm) KinE(MeV)  dE(MeV) StepLeng TrackLeng  NextVolume ProcName
    0        0        0    -3.64      2.05        0        0         0    Absorber initStep
    1 -0.000201 0.000128    -3.64      2.01   0.0491 0.000266  0.000266    Absorber ionIoni
    2 -0.00049 0.000312    -3.64      1.93   0.0705 0.000381  0.000647    Absorber ionIoni

*********************************************************************************************************
* G4Track Information:   Particle = triton,   Track ID = 2,   Parent ID = 1
*********************************************************************************************************

Step#    X(mm)    Y(mm)    Z(mm) KinE(MeV)  dE(MeV) StepLeng TrackLeng  NextVolume ProcName
    0        0        0    -3.64      2.73        0        0         0    Absorber initStep
    1 0.000339 -0.000215    -3.64      2.71   0.0116 0.000447  0.000447    Absorber hIoni
\end{lstlisting}


