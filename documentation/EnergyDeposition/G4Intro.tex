%%%%%%%%%%%%%%%%%%%%%%%%%%%%%%%%%%%%%
\section{Introduction to GEANT4}
\label{sec:G4Intro}

GEANT4 (GEomentry ANd Tracking) is a free, open source Monte Carlo based physics simulation toolkit developed and maintained at CERN widely used in the physics community.
A GEANT4 simulation starts with a run which contains a set number of events.
An event is particular process of interest to the user, such as shooting a single particle at a detector. 
Typical usage might be to have a run be firing 1,000 neutrons at a detector, were each neutron is a single event.

\subsubsection{Organization of the GEANT4 Toolkit}
The GEANT4 toolkit is divided into eight (8) class categories:
\begin{itemize}
    \item Run and Event - generation of events and secondary particles.
    \item Tracking and Track - transport of a particle by analyzing the factors limiting the step size and by applying the relevant physics models.
    \item Geometry and Magnetic Field - the geometrical definition of a detector (including the computation of the distances to solids) as wells as the management of magnetic fields.
    \item Particle Definition and Matter - definition of particles and matter.
    \item Hits and Digitization - the creation of hits and their use for digitization in order to model a detectors readout response.
    \item Visualization - the visualization of a simulation including the solid geometry, trajectories and hits.
    \item Interface - the interactions between the toolkit and graphical user interfaces and well as external software.
\end{itemize}

There are then three classes which must be implemented by the user in order use the toolkit. These classes are:
\begin{itemize}
    \item \verb+G4VuserDetectorConstruction+ which defines the geometry of the simulation,
    \item \verb+G4VUserPhysicsList+ which defines the physics of the simulation, and
    \item \verb+G4VUserPriamryGeneratorAction+ which defines the generation of primary events.
\end{itemize}
Five additional classes are available for further control over the simulation:
\begin{itemize}
    \item \verb+G4UserRunAction+ which allows for user actions
\end{itemize}
