\section{Introduction}
% Generic Background Paragraph

A typical day in 2011 saw 932,456 people enter into the U.S. (258,191 by air 48,073 by sea, and 621,874 by land) in addition to 64,483 truck, rail and sea containers and 253,821 privately-owned vehicles \cite{cpb_typical_2012}.
Any one of these could be a pathway of special nuclear material to enter the U.S.
The interdiction of special nuclear material is desirable before the materials enters into the transportation infrastructure of the U.S. and interdiction becomes more complex.
Radiation Portal Monitors (RPMs) are passive radiation detection systems implemented at over a thousand border crossings designed to determine if cargo contains any nuclear material in a safe, nondestructive and effective manner\cite{kouzes_neutron_2010}.
The Department of Homeland Security (DHS) continues to fund research through the Domestic Nuclear Detection Office (DNDO) in order to develop replacement technologies for the current \iso{He}{3} RPMs as \iso{He}{3} cannot be economically replaced.
There are several alternatives to \iso{He}{3} being considered, and all, with the exception of gas filled proportional detectors, involve the detection of neutrons from scintillation events of the energy deposited in the material from the neutron absorption reaction.
These detectors (among other requirements outline in Table \ref{tab:DHSCriteria}) must be able to effectively discriminate between gamma (which can occur in medical isotopes) and neutrons (indictive of special nuclear material).
\begin{table}[h]
    \caption{Replacement Detector Requirements \protect\cite{kouzes_neutron_1999}}
	\centering
	\begin{tabular}{p{0.4\textwidth} | p{0.4\textwidth} }
	Parameter & Specification \\
	\hline
	\hline
	Absolute neutron detection efficiency & 2.5 cps/ng of ${}^{252}$Cf \\
	Intrinsic gamma-neutron detection efficiency & $ \epsilon_{int,\gamma n}\leq 10^{-6}$ \\
	Gamma absolute rejection ratio for neutrons (GARRn) & $ 0.9 \leq \text{ GARRn }\leq$ 1.1 at 10 mR/h exposure \\
	Cost &  \$ 30,000 per system \\
	\hline
	\end{tabular}
    \label{tab:DHSCriteria}
\end{table}

% Introduction to charged particles / energy deposition
Neutron detectors often utilize a material doped with an isotope of large thermal cross section for absorption such as \iso{Li}{6} or \iso{B}{10}. 
When these materials absorb a neutron the nucleus of the isotope becomes unstable and fissions into reaction products.
These reaction products (having an initial kinetic energy from the Q-value of the neutron absorption reaction) travel through the material, transferring their kinetic energy to the material.
Photon interactions in the detector occur when a photon scatters off a single electron in a Compton scattering event (Table \ref{tab:ComptonScattering}).
This Compton electron then produces a cascade of secondary electrons in the material, which, depending upon the energy, may or may not deposit a majority of its energy in the detector.
The difference in the transfer of kinetic energy from charged particle to electrons and from photon interactions (Compton scattering) to electrons introduces an opportunity to exploit the difference in energy deposition in order to maximize the discrimination between neutron and photon interactions in a detector.
\begin{table}[hb]
	\centering
    \caption{Maximum Energy of Secondary Electrons from Compton Scattering}
	\begin{tabular}{c | c c }
	& Photon Energy (MeV) & Maximum Compton Energy (MeV) \\
	\hline
	\hline
    \iso{Cs}{137} & 0.662 & 0.478 \\
    \iso{Co}{60} & 1.17, 1.33 & 0.960, 1160 \\
	\hline
	\end{tabular}
    \label{tab:ComptonScattering}
\end{table}


This document is organized as follows.
A brief overview of the interaction of charged particles in matter will be provided in Section \ref{sec:PreviousWork}, as well as some preliminary experiments demonstrating the range of secondary electrons in neutron-gamma discrimination.
The GEANT4 toolkit was used for the modeling of the energy deposition.  Section \ref{sec:G4Intro} will provide an overview of the GEANT4 toolkit.
Section \ref{sec:Methods} will provide details on how the GEANT4 toolkit was implemented for this particular simulation, as well as providing validation of the calculations performed by the GEANT4 toolkit in Section \ref{sec:SimValidation}.
In Section \ref{sec:Results} the results of this model applied to a single film will demonstrate the enhanced ability of neutron-gamma discrimination through secondary electrons.
