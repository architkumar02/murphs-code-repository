\section{Results}
\label{sec:Results}

\subsection{Energy Deposition}
The energy deposition was calcutated for neutron and gamma events for films of thickness of 15 \micron, 25 \micron, 50 \micron, 150 \micron, 300 \micron, 600 \micron, 1 mm and 1 cm (Figure \ref{fig:SimEDepGamma}, \ref{fig:SimEDepNeutron}).  

Photons have a very low probability of interacting in the film due to polymer film being a low z-material.
This is reflected in the majority of the events not interacting at all; about 1 in 10,000 of the events deposit energy in the film as seen in Figure \ref{fig:SimEDepGamma}.
Several classic features of the spectra are apparent on the 1 cm thick thin.
These included the photo-peak in which all of the incident energy of the \iso{Co}{60} is deposited in the film, as well as the individual Compton edges of the two photons fromn \iso{Co}{60}.
These features are not visiable on the measured spectra due to the poor energy resolution of these films.
There is also physical evidance of a lack of a Compton edge on the thinner films, but the films greater than 150 \micron thick show some feature around 0.2 MeV.
Films thinner than 150 \micron show a very small amount of energy deposition that quickly tails off for higher energies, indicating that when a photon interaction occurs in the film the electrons from that interaction leave the film and the only energy deposition occurs from small ionizations as the highly energetic electron leaves the film material.
It is also observed that the thinnest film (15 \micron) has an average energy depostion of around 10 keV, while the 1 cm film has an average energy deposition of around 150 keV.
\begin{figure}[h]
    \includegraphics[width=\figurewidth]{Co60EnergyDep}
	\caption{Simulated Energy Depositon for a Single Film (gammas)}
    \label{fig:SimEDepGamma}
\end{figure}
The simulated energy deposition for neutron interactions in thin films is shown in Figure \ref{fig:SimEDepNeutron}.
Several features of the spectra can be immediately noted.
For thick films (1 cm) there is a very high probability that a given event will deposit all of its energy in the film (as expected).
Thinner films have a smaller probability of depositing all of their energy, but this is overshawded by the thick samples when plotted.
It is also intresting to note that it is possible to observe the comparative effects of the the $\alpha$ and \iso{H}{3} in the neutron energy depostion spectra. 
The triton has a much shorter range (\~ 10 \micron in PS \cite{kudo_recoil_1980}) than the $\alpha$ (\~ 60 \micron) so it has a higher probability of depositing all of its energy.
Thus, for energies above 2.73 MeV (the energy of the triton) there is a higher probability of energy energy deposition (by about a factor of 10). These events are still very infrequent compared to the probability of depositing all of the reaction product energy.
Even for the 15 \micron the average energy depostion was above 50\% of the total Q-value of the reaction, and by 200 \micron this average energy deposited approaches 95\% of the total 4.78 MeV.
\begin{figure}[h]
    \includegraphics[width=\figurewidth]{NeutronEnergyDep}
	\caption{Simulated Energy Depositon for a Single Film (neutrons)}
    \label{fig:SimEDepNeutron}
\end{figure}
The average energy deposited was computed for each thickness and normalized by the incident energy for gammas by the Q-value of the reaction for neutrons, and is presented in Table ~\ref{tab:FractionEDep}.
For thickness greater than \SI{150}{\um} there is little benefit in increasing the thickness of the film in terms of energy deposition by neutrons, since over 90\% of the energy is being deposited in the film.
A comparision between the energy deposition of neutrons and gamma's is shown for various film thickness in Figure \ref{fig:SimEDepNGComparison}.
\begin{table}[ht]
    \caption{Fractional Energy Deposition for Various Thickness}
	\centering
	\begin{tabular}{c | c c}
	Thickness & Gamma Fraction & Neutron Fraction \\
	\hline
	\hline
	\SI{15}{\um} & 0.010 & 0.531 \\
	\SI{25}{\um} & 0.013 & 0.634 \\
	\SI{50}{\um} & 0.017 & 0.782 \\
	\SI{150}{\um} & 0.032 & 0.927 \\
	\SI{300}{\um} & 0.052 & 0.964 \\
	\SI{600}{\um} & 0.087 & 0.982 \\
	\SI{1}{\mm} & 0.130 & 0.989 \\
	\SI{1}{\cm} & 0.425 & 0.998 \\
	\end{tabular}
  \label{tab:FractionEDep}
\end{table}
\begin{figure}
  \includegraphics[width=\figurewidth]{EDepSummary}
  \caption[Comparison of Neutron and Gamma Energy Depostion]{GEANT4 simulated energy deposition for neutrons and gamma's as a function of film thickness.  It is observed in both cases that energy deposition exponentially approaches it's maximum value}
  \label{fig:SimEDepNGComparison}
\end{figure}

\subsection{Positional Dependance of Energy Deposition}
A neutron or gamma event that takes place near the surface of the film has a higher probability of depsoiting less than the maximum amount of energy due to the possibility of the escape of secondary particles through the surface of the film compared to the volume of the film, which goes as $\frac{1}{r}$\footnote{%%
The surface area to volume ratio of the films is $\frac{SA_\text{cyl}}{\forall_\text{cyl}} = \frac{2\pi r^2+2\pi rh}{\pi r^2 h}$, which reduces to $\frac{2(r +h)}{rh}$. 
If it is assumed that the radius is much smaller than the height of the cylinder, than $\frac{SA_\text{cyl}}{\forall_\text{cyl}} = \frac{2}{r}$.
}
Thefore, the probability of depositing energy as a function of thickness through the film was investigated by creating a histogram on the first interaction position and the corresponding amount of energy that was deposited.

It is observed for gamma interactions that for films thinner than \SI{1}{\cm} the energy deposited decrease rapidly as the interaction takes placer deeper in the film.
However, this effect is diminished for films that are thick enough to have a high probability of a gamma interaction occuring in the middle of the film where there is enough material to still collect the energy.
\todo{Need to switch around the gamma's maybe a contour plot?}
\todo[inline]{Do we see the effects of electrons being back-scattered from the arcylic disk in the gamma?}
% The 0.1 cm neturon has a nice horseshoe shape indicative of the features of intrest.
\subsection{Secondary Electron Energy Distribuion}
Figure \ref{fig:simKinE} illustrates the simulated kinetic energy of secondary electrons from Compton scattering and from alpha and triton interactions.
It is observed that kinetic energy of the secondary electrons from the neutron reaction products have predominately energies in the kilo-volt range, while the Compton scattering electrons have energies in hundreds of kilo-volts range. 
However, it should be noted that there is only one secondary electron from a Compton scattering and multiple secondary electrons from the reaction products.
Figure \ref{fig:ReacProdDist} shows the distribution of the number of secondary electrons from the alpha and triton and their kinetic energy.
\begin{figure}[ht]
    \centering
    \includegraphics[width=\textwidth]{NGSecElecKinEDist}
    \caption{Simulated Kinetic Energy of Secondary Electrons from Compton Scattering and from \iso[6]{Li} reaction products}
    \label{fig:simKinE}
\end{figure}
\begin{figure*}[ht]
	\centering
	\begin{subfigure}[b]{0.45\textwidth}
    		\includegraphics[width=\textwidth]{AlphaTritonSecElecKinEDist}
		\caption{Alpha and Triton Secondary Electron Kinetic Energy Distribution}
	\end{subfigure}%
	~
	\begin{subfigure}[b]{0.45\textwidth}
    		\includegraphics[width=\textwidth]{NeutronNumSecElec}
		\caption{Number of Secondary Electrons Produced Per Neutron Interaction}
	\end{subfigure}%
	\caption{Neutron Reaction Products Secondary Electrons Energies}
	\label{fig:ReacProdDist}
\end{figure*}

The distribution of secondary electrons from photon interactions are plotted in Figure \ref{fig:SecElecKinEDist}.
From these results it can be concluded that the it is unlikely (around 1 in 10,000) that an electron will be scattered with the maximum Compton scattering kinetic energy, but rather have an energy somewhat lower than that.
The distribution of secondary electrons from photon interactions is actually very flat, implying that it is likely for the electron from a Compton scattering event to have an energy in the 100's of keV.
The distribution of the next generation of electrons was also calculated, and this distrubiton was also quite entergetic (with a maximum energy corresponding to 0.55 MeV) but with a much large probability of having a collision that produces and electron with a much lower energy.
\begin{figure}[h]
    \centering
    \begin{subfigure}[b]{0.45\figurewidth}
        \includegraphics[width=\figurewidth]{Co60_PID1_SecElectron}
        \caption{First Secondary Electron}
    \end{subfigure}
    \begin{subfigure}[b]{0.45\figurewidth}
        \includegraphics[width=\figurewidth]{Co60_PID2_SecElectron}
        \caption{Second Secondary Electron}
    \end{subfigure}
    \caption{Simulated kinetic energies of electrons from \iso{Co}{60} interactions}
    \label{fig:SecElecKinEDist}
\end{figure}
\todo[inline]{A verification of sorts was completed by looking at the compton scatted spectra.  I think that I would like to put it as an appendix to my thesis}
