\documentclass[draftcls,onecolumn]{IEEEtran}

%% INCLUDING THE PREAMBLE
%%%%%%%%%%%%%%%%%%%%%%%%%%%%%%%%%%%%%%%%%%%%%%%%%%%%%%%%%%%%%%%%%%%%%%%%%%%
%                                                                         %
%                                 PREAMBLE                                %
%                                                                         %
%%%%%%%%%%%%%%%%%%%%%%%%%%%%%%%%%%%%%%%%%%%%%%%%%%%%%%%%%%%%%%%%%%%%%%%%%%%

%% PACKAGES
\usepackage[]{lineno}
%\linenumbers
\usepackage[usenames,dvipsnames]{xcolor}
\usepackage{microtype}
\usepackage[obeyDraft]{todonotes}
\usepackage{fancyvrb}
\VerbatimFootnotes
\usepackage{algorithmic}

%% GRAPHICS RELATED
\usepackage{graphicx}
\usepackage[outdir=./tmp/]{epstopdf}
\graphicspath{{../images/}{./}{./tmp/}}
\DeclareGraphicsExtensions{.eps, .pdf, .jpeg, .png,}

%% CPATION SETUP
\usepackage{float}
\usepackage{caption}
\usepackage{subcaption}
\captionsetup{belowskip=12pt,aboveskip=4pt}


%% BIBLIOGRAPHY
\bibliographystyle{ieeetr}

%% UNITS
\usepackage{siunitx}

%% EQUATIONS
\usepackage{amsmath}
%\numberwithin{equation}{section}

%% HYPERLINKS
\usepackage[debug]{hyperref}

%%%%%%%%%%%%%%%%%%%%%%%%%%%%%%%%%%%%%%%%%%%%%%%%%%%%%%%%%%%%%%%%%%%%%%%%%%%
%                                                                         %
%                             Listing Setup                               %
%                                                                         %
%%%%%%%%%%%%%%%%%%%%%%%%%%%%%%%%%%%%%%%%%%%%%%%%%%%%%%%%%%%%%%%%%%%%%%%%%%%
\usepackage{listings}
\lstset{ %
    language=C++,
    basicstyle=\footnotesize\ttfamily,
    numbers=left,
    numberstyle=\tiny\color{gray},
    stepnumber=2,
    numbersep=5pt,
    backgroundcolor=\color{white},
    showspaces=false,
    showstringspaces=false,
    showtabs=false,
    frame=single,
    rulecolor=\color{black},
    tabsize=2,
    breaklines=true,
    breakatwhitespace=false,
    title=\lstname,
    keywordstyle=\color{blue},
    commentstyle=\color{OliveGreen},
    stringstyle=\color{orange}
}
\DeclareCaptionFont{white}{\color{white}}
\DeclareCaptionFormat{listing}{\colorbox[cmyk]{0.43, 0.35, 0.35, 0.01}{\parbox{\dimexpr\textwidth-2\fboxsep\relax}{#1#2#3}}}
\captionsetup[lstlisting]{format=listing,labelfont=white,textfont=white,singlelinecheck=false,margin=0pt,font={bf,footnotesize}}
%\lstnewenvironment{code}[1][]%
%{ \noindent\minipage{\linewidth}
%	\lstset{#1}
%}
%{\endminipage}
%% USER COMMANDS
\usepackage{isotope}
\newcommand{\iso}{\isotope}
\newcommand{\figurewidth}{\textwidth}
\newcommand{\micron}{$\mu$m}



%%%%%%%%%%%%%%%%%%%%%%%%%%%%%%%%%%%%%%%%%%%%%%%%%%%%%%%%%%%%%%%%%%%%%%%%%%%
%                                                                         %
%                                Start of Document                        %
%                                                                         %
%%%%%%%%%%%%%%%%%%%%%%%%%%%%%%%%%%%%%%%%%%%%%%%%%%%%%%%%%%%%%%%%%%%%%%%%%%%
\begin{document}
\title{Layered Detectors}
\author{Matthew J. Urffer}
\date{\today}
\maketitle


% Tables of Contents, Figures, Tables
\listoftodos
\tableofcontents
\listoffigures
\listoftables
\lstlistoflistings
%%%%%%%%%%%%%%%%%%%%%%%%%%%%%%%%%%%%%%%%%%%%%%%%%%%%%%%%%%%%%%%%%%%%%%%%%%%
%                                                                         %
%                              Start of Content                           %
%                                                                         %
%%%%%%%%%%%%%%%%%%%%%%%%%%%%%%%%%%%%%%%%%%%%%%%%%%%%%%%%%%%%%%%%%%%%%%%%%%%
\section{Introduction}
It was observed in repeating the neturon fraction above the gamma LLD that there was signficant disgrepancy between a film mounted on a thin PMMA disc and a film mounted on a thick UVT (ultra-violet tranpsarent) aryclic disc.

Several differnet causes of this disgregency were considered:
\begin{itemize}
  \item energy deposition by charged particles not being fully absorbed in the thin disc,
  \item light propogation through the aryclic disc being attenuated,
  \item and adverse chemical reactions in the PMMA disc.
\end{item}

It was quickly ruled out that the energy deposition by the reaction products could not be repsonsible.
As both materials are made of the same constituents it is reasonable to expect that the reflection of electron from the disc back unto the film would be very similar.
\todo[inline]{I could verify this with a quick GEANT4 sim, changing the thickness of the backing}.
In addition, a \iso[36]{Cl} source was placed directly on the PMT and \todo{do this} and an insignifcant count rate was observed, thus making it unlikely that electrons were escaping the film and interacting directly with the photo-diode on the PMT.

\section{Methods}
Two different experiments were preformed in order to determine the cause of this.
The first experiment was to try and reconstruct the same geometry for a thin disc as if it was being measured as a thick disc.
The second (after the first failure to provide a conclusive answer) was to fabricate two more films of the same composition and batch, measuring the emission and excitation of the disc before and after, along with the scintillation properties.

\subsection{Light Transport Experiment}

\subsection{Film Fabrication Experiment}

\section{Results}

\end{document}
