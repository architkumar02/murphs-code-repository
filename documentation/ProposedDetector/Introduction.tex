\section{Introduction}
It is proposed to build a detector in order to verify the GEANT4 simulations while providing a framework to test light transport collection, spacing, and material properties in order to build a scaled modeled of a layered RPM8 detector.
There are several design parameters which must be determined:
\begin{itemize}
    \item detector material and thickness,
    \item detector encapsulating material and thickness,
    \item detector encapsulating material thickness,
    \item and spacing between layers.
\end{itemize}
\begin{figure}
    \centering
    \includegraphics[]{ThinFilmDetectorDesign}
    \caption{Detector Design Parameters}
    \label{fig:DetectorParameterSchematic}
\end{figure}
Figure ~\ref{fig:DetectorParameterSchematic} provides a schematic of the geometry discused.
\subsection{Detector Material}
${}^6$LiF was chosen as the material of choice because of it's high light output ($16x10^4$ photons per neutron) and large ${}^{6}$LiF content \cite{carel_w.e_inorganic-scintillator_2001}. 
Innovative American Technologies (IAT) have pursued ${}^6$LiF:ZnS based detectors with mixed results.
The IAT detector is a LiF:ZnS coated paddle with two PMT's at either end of an assembly that is 12" x 5" x 85" (size of RPM8)~\ref{fig:IATPaddle}.
The neutron efficiency was reported by the manufacture to be 3.7 cps per ng ${}^{252}$Cf, with a gamma sensitivity of less than $10^{-7}$ in a $>$ 20mR per hr gamma field.  The GARRn was within the criteria.
Previous models tested at PNNL in 2010 were scaled models which did not have an adequate neutron count rate but passed the other criteria \cite{kouzes_lithium_2010}.
\begin{figure}
    \centering
    \includegraphics[width=\textwidth]{IAT_NDM_4-0}
    \caption{IAT NDM 4.0 ${}^6$LiF:ZnS neutron detector (12" x 5" x 85")}
    \label{fig:IATPaddle}
\end{figure}

Eljen Technology (Sweetwater, TX) provides two ${}^6$LiF:ZnS screens, the EJ-426-0 and EJ-426-HD2 with backing materials   of 50 $\mu$m of aluminum foil, 0.12 mm of aluminized mylar, a 0.5 mm aluminum sheet, a 0.4 mm sheet of highly reflective aluminum, and 0.25 mm of clear polyester with the option of a 1 mm acrylic plate \cite{eljen_ej-426_2012}.
It was decided to use EJ-426-HD2 because it contains 1.6 times the ${}^{6}$Li content as EJ-426-0. EJ-426 has a ${}^{6}$Li atomic density of $1.39\times10^{22}$ atoms of ${}^6$Li per cm${}^3$ \cite{eljen_ej-426_2012}. 
The material composition of EJ-425-HD2 is in shown in Table ~\ref{tab:EJ426Composition}, with a density of 4.1 g/cm${}^3$\cite{carel_w.e_inorganic-scintillator_2001}.
EJ-426 is opaque, however, due to the ZnS.  In addition previous experience with EJ-426 a slight brittleness was observed.
\begin{table}
    \centering
    \caption[Caption for LOF]{EJ-426-HD2 Composition\protect\footnotemark \cite{urffer_ej_2012} }
    \begin{tabular}{c|c c}
        Isotope & Atomic Density ($\times 10^{22}$ atoms per cm${}^3$) & Mass Fraction\\
        \hline
        $ {}^6 $ Li & 1.38 & 0.076\\
        $ {}^7 $ Li & 0.073 & 0.005\\
        F & 1.46 & 0.253\\
        Zn & 0.750 & 0.447\\
        S & 0.750 & 0.219\\
    \end{tabular}
    \label{tab:EJ426Composition}
\end{table}
    \footnotetext{Not listed is the proprietary binder which is less than 15\% of the total composition}
The detector thickness is a key design parameter. 
Thick detector will have a high neutron efficiency with the large amount of neutron absorber, but due to LiF:ZnS being opaque neutron interactions that produce light in the middle of the film will not propagate their light outside of the film.
\todo{What is the self absorption (light) of EJ-426?}
\missingfigure{Range of alpha,triton of EJ-426}

\subsection{Detector Encapsulating Material}
It is proposed to wrap the detector material with an encapsulating material.
Two classes of materials exists; scintillating materials and non-scintillating materials.  
Scintillating materials have the ability to capture the energy that escapes from the detector, which is desirable for neutron interactions but not for gammas.

