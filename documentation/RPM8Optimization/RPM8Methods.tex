\section{Methods}
\label{sec:Methodes}


The optimization of the films was formulated as finding the minimum mass of \iso[6]{Li} for a given detector material necesary to fulfilll an interaction rate of \SI{2.5}{cps\per\nano\gram\iso[252]{Cf}} while mainting a intrisinic gamma rejection ratio of \num{1.e-6}.

\subsection{Design Parameters}
\label{sec:DesignParameters}
The design parameters were then:
\begin{itemize}
  \item The detector material. Materials with a high \iso{6}[Li] concentration will observe more neturons.
  \item The thickness of the detector material.
  \item The spacing of detector layers.
  \item The initial moderator thickness.
\end{itemize}

The intial moderator thickness was set at \SI{2.5}{\centi \meter}.
At this thickness around 10\% of the neutrons are thermalized (\todo{Cite my report})\footnote{The thermal energy was chosen to be \SI{5}{\ev}}.

The optimization of the detector design is presented in two parts: a numerical approcach in which a matrix of detector spacing and assemblies are varied (Section ~\ref{sec:MCNPXMethods}, and a multi-variate optimization in which the paramters are fit to a functional form which is then optimized (Section ~\ref{sec:MVOptimization}.
\subsection{MCNPX Simulations}
\label{sec:MCNPXMethods}
A matrix of detector designs was simulated in MCNPX.
A generic script file was writen (Listing ~\ref{lst:MCNPXScript}) that was modified to include a user supplied number of detector assemblies, spacing between these assemblies, and initial moderator at the front of the assembly.


\subsection{1 D Transport Part}
The large detector and far away source suggest that the problem can be simplified into a one dimenionsal transport problem without suffering the accuracy of the solution.
As 1D deterministic transport is much faster than 3D Monte Carlo, 1D transport was explored in order to vary the problem parameters.

\subsection{Multi-variate Optimization}
\label{sec:MVOptimization}
