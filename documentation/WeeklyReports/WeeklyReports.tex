%%%%%%%%%%%%%%%%%%%%%%%%%%%%%%%%%%%%%%%%%%%%%%%%%%%%%%%%%%%%%%%%%%%%%%%%%%%%%%%
%                                                                             %
%                                Weekly Reports                               %
%                                                                             %
%%%%%%%%%%%%%%%%%%%%%%%%%%%%%%%%%%%%%%%%%%%%%%%%%%%%%%%%%%%%%%%%%%%%%%%%%%%%%%%

%----------------------------------------------------------------------------------------
%	PACKAGES AND OTHER DOCUMENT CONFIGURATIONS
%----------------------------------------------------------------------------------------
\documentclass[paper=a4,twoside,captions=tableheading,index=totoc,hyperref]{labbook}

\usepackage[english]{babel}   % English language
\usepackage{lipsum}           % Used for inserting dummy 'Lorem ipsum' text into the template
\usepackage[utf8]{inputenc}   % Uses the utf8 input encoding
\usepackage[T1]{fontenc}      % Use 8-bit encoding that has 256 glyphs

\usepackage{caption}
\usepackage{booktabs,array}   % Packages for tables
\bibliographystyle{ieeetr}
\usepackage{etoolbox}
\usepackage{lastpage}

%% GRAPHICS and IMAGES
\usepackage{graphicx}
\usepackage[usenames,dvipsnames]{xcolor}
\usepackage[outdir=./tmp/]{epstopdf}
\graphicspath{{../images/}{./}{./tmp/}}
\DeclareGraphicsExtensions{.eps, .pdf, .jpeg, .png,}

% Lab book setup
\usepackage[nouppercase,headsepline]{scrpage2}
\pagestyle{scrheadings}
\clearscrheadfoot
\automark[chapter]{chapter}
\ohead{\headmark}
\ofoot[\normalfont\normalcolor{\thepage\ |\  \pageref{LastPage}}]{\normalfont\normalcolor{\thepage\ |\  \pageref{LastPage}}}
\makeatletter
\patchcmd{\addchap}{\if@openright\cleardoublepage\else\clearpage\fi}{\par}{}{}
\makeatother
\renewcommand*{\chapterpagestyle}{scrheadings}
\usepackage{chngcntr}
\counterwithout{figure}{labday}
\counterwithout{equation}{labday}


%% NOMENCLATURE
\usepackage[refpage]{nomencl}  % refer to the page where notation appears
\newcommand{\definevar}[2]{#1 is the #2\nomenclature{#1}{#2}}
\newcommand{\nom}[2]{#1 #2\nomenclature{#1}{#2}}
\renewcommand{\nomname}{List of Notations}
\renewcommand*{\pagedeclaration}[1]{\unskip\dotfill\hyperpage{#1}}
\makenomenclature
\usepackage{listings}
\lstset{ %
    language=C++,
    basicstyle=\footnotesize\ttfamily,
    numbers=left,
    numberstyle=\tiny\color{gray},
    stepnumber=2,
    numbersep=5pt,
    backgroundcolor=\color{white},
    showspaces=false,
    showstringspaces=false,
    showtabs=false,
    frame=single,
    rulecolor=\color{black},
    tabsize=2,
    breaklines=true,
    breakatwhitespace=false,
    title=\lstname,
    keywordstyle=\color{blue},
    commentstyle=\color{OliveGreen},
    stringstyle=\color{orange}
}
\DeclareCaptionFont{white}{\color{white}}
\DeclareCaptionFormat{listing}{\colorbox[cmyk]{0.43, 0.35, 0.35, 0.01}{\parbox{\dimexpr\textwidth-2\fboxsep\relax}{#1#2#3}}}
\captionsetup[lstlisting]{format=listing,labelfont=white,textfont=white,singlelinecheck=false,margin=0pt,font={bf,footnotesize}}

%% MATH / SCIENCE Packages
\usepackage{algorithmic}
\usepackage{siunitx}
\usepackage{amsmath}
\usepackage{isotope}
\newcommand{\iso}{\isotope}
\usepackage{todonotes}

% Hyperlink configuration
\usepackage[
    pdfauthor={Matthew J. Urffer},
    pdftitle={Laboratory Journal},
    pdfsubject={},
    bookmarksopen=true,
    linktocpage=true,
    backref=page,
    pdfpagelabels=true,
    plainpages=false,
    colorlinks=true,
    bookmarks=true]{hyperref}
%----------------------------------------------------------------------------------------
%	DEFINITION OF EXPERIMENTS
%----------------------------------------------------------------------------------------

% Template: \newexperiment{<abbrev>}[<short form>]{<long form>}
% <abbrev> is the reference to use later in the .tex file in \experiment{}, the <short form> is only used in the table of contents and running title - it is optional, <long form> is what is printed in the lab book itself

\newexperiment{selfsheild}[GS20 Self Sheilding]{Self Sheilding in GS20}
\newexperiment{uvtpmma}[UVT Arcylic and PMMA Aryclic]{Effects of UVT Arcylic and PMMA Discs}
\newexperiment{characterization}[Detector Characterization]{Detector Characterization}
\newexperiment{simulation}[MCNPX Simulations]{Monte Carlo Detector Simulations}

\newsubexperiment{psrepeatibilty}[PS Film Repeatiblity]{Polystyrene Based Film Repetibliyt}
\newsubexperiment{detectorcomp}[Detector Comparison]{Comparison of Characterized Detectors}

%----------------------------------------------------------------------------------------

\begin{document}

%----------------------------------------------------------------------------------------
%	TITLE PAGE
%----------------------------------------------------------------------------------------
\title{Weekly Reports}
\author{Matthew J. Urffer}
\date{\today}
\maketitle

\listoftodos
\printnomenclature
\printindex
\tableofcontents
\listoffigures
\listoftables
\lstlistoflistings

\newpage % Start lab look on a new page

\pagestyle{scrheadings} % Begin using headers

%----------------------------------------------------------------------------------------
%	LAB BOOK CONTENTS
%----------------------------------------------------------------------------------------
\labday{April 29, 2013}

\subexperiment{detectorcomp}
The mass fractions of GS20, EJ-426 HD2, and W
\labday{Week of 4/16/2013}
Weekly progress:

\lipsum[1]

%-----------------------------------------

\experiment{example}

\lipsum[2]

Footnote example\footnote{Lorem ipsum dolor sit amet, consectetuer adipiscing elit.}.\\

Citation example \cite{lamport94}.

%-----------------------------------------

\subexperiment{subexp_example}

\lipsum[3]

\begin{table}
\label{tab:treatments_xy}
\raggedleft
\begin{tabular}{l l l}
\toprule
\textbf{Groups} & \textbf{Treatment X} & \textbf{Treatment Y} \\
\toprule
1 & 0.2 & 0.8\\
2 & 0.17 & 0.7\\
3 & 0.24 & 0.75\\
4 & 0.68 & 0.3\\
\bottomrule\\
\end{tabular}
\caption{The effects of treatments X and Y on the four groups studied.}
\end{table}

%-----------------------------------------

\subexperiment{subexp_example2}

\lipsum[4]

\begin{figure}[h!]
\raggedleft
\includegraphics[scale=0.5,keepaspectratio=true]{placeholder.png}
\caption{Placeholder image.}
\label{fig:placeholder}
\end{figure}

\lipsum[5]

Example figure citation: \autoref{fig:placeholder}.

%-----------------------------------------

\subexperiment{subexp_example3}

\lipsum[6]

%-----------------------------------------

\experiment{example2}

\lipsum[7]

%-----------------------------------------

\subexperiment{subexp_example}

\lipsum[8]

%-----------------------------------------

\experiment{example3}

\subexperiment{subexp_example}
  
\lipsum[9]

%----------------------------------------------------------------------------------------

\labday{Saturday, 2 June 2012}

\experiment{example}

\lipsum[10]

\begin{equation}
\label{eq:emc}
e = mc^2
\end{equation}

Example equation citation: \autoref{eq:emc}.

%-----------------------------------------

\subexperiment{subexp_example}

\lipsum[11-13]

%-----------------------------------------

\experiment{example2}

\lipsum[14]

%-----------------------------------------

\subexperiment{subexp_example}

\lipsum[15]
 
 

%----------------------------------------------------------------------------------------
%	BIBLIOGRAPHY
%----------------------------------------------------------------------------------------

\begin{thebibliography}{9}

\bibitem{lamport94}
Leslie Lamport,
\emph{\LaTeX: A Document Preparation System}.
Addison Wesley, Massachusetts,
2nd Edition,
1994.

\end{thebibliography}

%----------------------------------------------------------------------------------------

\end{document}
