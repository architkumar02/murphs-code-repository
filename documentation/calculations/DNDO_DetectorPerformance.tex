% Intrinsic Efficiency
\documentclass[draftcls,onecolumn]{IEEEtran}

%% INCLUDING THE PREAMBLE
%%%%%%%%%%%%%%%%%%%%%%%%%%%%%%%%%%%%%%%%%%%%%%%%%%%%%%%%%%%%%%%%%%%%%%%%%%%
%                                                                         %
%                                 PREAMBLE                                %
%                                                                         %
%%%%%%%%%%%%%%%%%%%%%%%%%%%%%%%%%%%%%%%%%%%%%%%%%%%%%%%%%%%%%%%%%%%%%%%%%%%

%% PACKAGES
\usepackage[]{lineno}
%\linenumbers
\usepackage[usenames,dvipsnames]{xcolor}
\usepackage{microtype}
\usepackage[obeyDraft]{todonotes}
\usepackage{fancyvrb}
\VerbatimFootnotes
\usepackage{algorithmic}

%% GRAPHICS RELATED
\usepackage{graphicx}
\usepackage[outdir=./tmp/]{epstopdf}
\graphicspath{{../images/}{./}{./tmp/}}
\DeclareGraphicsExtensions{.eps, .pdf, .jpeg, .png,}

%% CPATION SETUP
\usepackage{float}
\usepackage{caption}
\usepackage{subcaption}
\captionsetup{belowskip=12pt,aboveskip=4pt}


%% BIBLIOGRAPHY
\bibliographystyle{ieeetr}

%% UNITS
\usepackage{siunitx}

%% EQUATIONS
\usepackage{amsmath}
%\numberwithin{equation}{section}

%% HYPERLINKS
\usepackage[debug]{hyperref}

%%%%%%%%%%%%%%%%%%%%%%%%%%%%%%%%%%%%%%%%%%%%%%%%%%%%%%%%%%%%%%%%%%%%%%%%%%%
%                                                                         %
%                             Listing Setup                               %
%                                                                         %
%%%%%%%%%%%%%%%%%%%%%%%%%%%%%%%%%%%%%%%%%%%%%%%%%%%%%%%%%%%%%%%%%%%%%%%%%%%
\usepackage{listings}
\lstset{ %
    language=C++,
    basicstyle=\footnotesize\ttfamily,
    numbers=left,
    numberstyle=\tiny\color{gray},
    stepnumber=2,
    numbersep=5pt,
    backgroundcolor=\color{white},
    showspaces=false,
    showstringspaces=false,
    showtabs=false,
    frame=single,
    rulecolor=\color{black},
    tabsize=2,
    breaklines=true,
    breakatwhitespace=false,
    title=\lstname,
    keywordstyle=\color{blue},
    commentstyle=\color{OliveGreen},
    stringstyle=\color{orange}
}
\DeclareCaptionFont{white}{\color{white}}
\DeclareCaptionFormat{listing}{\colorbox[cmyk]{0.43, 0.35, 0.35, 0.01}{\parbox{\dimexpr\textwidth-2\fboxsep\relax}{#1#2#3}}}
\captionsetup[lstlisting]{format=listing,labelfont=white,textfont=white,singlelinecheck=false,margin=0pt,font={bf,footnotesize}}
%\lstnewenvironment{code}[1][]%
%{ \noindent\minipage{\linewidth}
%	\lstset{#1}
%}
%{\endminipage}
%% USER COMMANDS
\usepackage{isotope}
\newcommand{\iso}{\isotope}
\newcommand{\figurewidth}{\textwidth}
\newcommand{\micron}{$\mu$m}


% index generation
\usepackage{makeidx}
\usepackage{exercise}
\makeindex
 
% 'list of notations' generation
\usepackage[refpage]{nomencl}  % refer to the page where notation appears
\newcommand{\definevar}[2]{#1 #2\nomenclature{#1}{#2}}
\renewcommand{\nomname}{List of Notations}
\renewcommand*{\pagedeclaration}[1]{\unskip\dotfill\hyperpage{#1}}
\makenomenclature
 


%% Start of the document
\begin{document}
\title{Simulation of DHS Detector Performance}
\author{Matthew J. Urffer}
\date{\today}
\maketitle

% Nomenclature
\printnomenclature
\printindex

% Tables of Contents, Figures, Tables
\listoftodos
\tableofcontents
\listoffigures
\listoftables
\lstlistoflistings
\section{Introduction}

The performance of films is simulated with in MCNPX, a Monte Carlo transport code\cite{pelowitz_mcnpx_????}.
The geometry is as in the PNNL reports, namely a nano-gram of \iso[252]{Cf}  encased in \SI{0.5}{\cm} of lead and \SI{2.5}{\cm} of HDPE. 
The size of the RPM8 is \SI{12.7}{\cm} deep, by \SI{30}{\cm} wide and \SI{2}{\m} tall.

\section{Count Rate Calculation}
The interaction rate is calculated using the a cell flux tally in MCNPX and a tally multiplier card.
The tally multiplier card (FMn) is used to calculated any quantity of the form \eqref{eqn:FMCardForm} \cite{pelowitz_mcnpx_????}.
\begin{align}
  \label{eqn:FMCardForm}
  I &= C\int\phi(E)\Re_m(E)dE
\end{align}
\nom{$I$}{Interaction rate}
\nom{$\phi(E)$}{Energy dependent fluence}
\nom{$\Re_m(E)$}{Response function operator}
where
\begin{itemize}
  \item [] $I$ is the interaction rate,
  \item [] $\phi(E)$ is the energy-dependent fluence (particles per cm squared) ,
  \item [] $\Re_m (E)$ is an linear operator of response functions, including cross section libraries, and
  \item [] $C$ is an arbitrary scalar for normalization.
\end{itemize}
An general example of the use of the FM card is shown in Listing \ref{lst:GeneralFMExample}, which is taken from the MCNP manual \cite{pelowitz_mcnpx_????}.
% See pg. 4-41 of the MCNP manual
\begin{lstlisting}[caption={[Example usage of the FM card]Example usage of the FM card to calculate the number of reactions per \si{\cm\cubed} of type R in cell 8 of material M. The normalization is by atomic density, signified by the -1},label{lst:GeneralFMExample}]
F104: N 8
FM104 -1 M R
\end{lstlisting}

\subsection{Neutron count rate calculation}
The reaction rate $\iso[7]{Li}\left(\text{n},\text{t}\right)\alpha$ can be calculated by then applying the appropriate input for the FMn card and using an F4 card to calculate $\phi(E)$.
It should be noted that depending on the form of the cell flux card it may be necessary to normalize by the volume of the cell, $\forall$.
\nom{$\forall$}{Volume of the cell}

\subsubsection{Neutron Irridiator}
\label{sec:NeutronIrridiatorInteractionRate}
This is shown in Listing \ref{lst:InteractionTallyExample}, where the reaction number is 105 and the material number of the detector is 3.
The cell volume is question is 601 inside universe 610 for the lead well, and cell 601 inside universe 620 for the cadmium well.
\begin{lstlisting}[caption={[Lead and Cadmium Well ${}^{6}\text{Li}\left(\text{n},\text{t}\right)\alpha$ Reaction Rate]Lead and Cadmium Well ${}^{6}\text{Li}\left(\text{n},\text{t}\right)\alpha$ Reaction Rates. The lead well is 154, and the cadmium well is 254},label={lst:InteractionTallyExample}]
  FC154 (n,t) Reactions in Detector in Pb Well
  F154:n (601<610)
  FM154 -1 3 105
  FC254 (n,t) Reactions in Detector in Cd Well
  F254:n (601<620)
  FM254 -1 3 105
\end{lstlisting}
The film's thermal interaction rate (the analog of the spectra subtraction of the lead and cadmium wells) is simply then the difference in the interaction rate of each well, normalize by the volume (if necessary) and the source strength on the date of measurment \eqref{eqn:MillerInteractionRate}.
\begin{align}
  I_{\text{sim}} &= S(t) \forall \left( C\int\phi_{\text{Pb}}(E)\Re_m(E)dE - C\int\phi_{\text{Cd}}(E)\Re_m(E)dE \right ) \notag \\
   &= S(t) \forall \left( I_{\text{Pb}} - I_{\text{Cd}}\right)
\end{align}
In \eqref{eqn:MillerInteractionRate} $S(t)$ is the time dependant source strength, and $I_{\text{Pb}}$ and $I_{\text{Cd}}$ are the MCNPX tallies as computed in Listing \ref{lst:InteractionTallyExample}.
The normalization of the volume is necessary in this calculation as an \verb+SD+ card was not used in both flux tallies.

\subsubsection{RPM DHS-DNDO Replacment Detector}

The interaction rate in a simulated RPM8 replacement detector is calculated in a similar manner as Section \ref{sec:NeutronIrridiatorInteractionRate}; the interaction rate as computed by the \verb+FMn+ is multiplied by the source strength and volume if necessary.
An example of the MCNPX input cards is shown in Listing \ref{lst:InteractionRateRPM}.
Given that there the thermal response is not desired, there is no need to subtract out the differneces between the spectra, and the interaction rate is simply \eqref{eqn:RPM8InteractionRate}.
Note that in this calculation the source strength is set to be \SI{1}{\nano\gram} \iso[252]{Cf}, which has a neutron emission rate of \SI{2.3E3}{neutron\per\second}.
This is in accordance with the direct evaluation of the PNNL critera, which require a absolute neutron count rate of \SI{2.5}{count\per\second\per\nano\gram\iso[252]{Cf}}.
\begin{lstlisting}[caption={[RPM8 ${}^{6}\text{Li}\left(\text{n},\text{t}\right)\alpha$ Reaction Rate]RPM8 ${}^{6}\text{Li}\left(\text{n},\text{t}\right)\alpha$ Reaction Rate. The detector is all of the layers of cell 500 inside universe 610. This tally is multiplied by an SD card to normalize by the volume},label={lst:InteractionRateRPM}]
FC4 (n,t) Reactions in Thin Film (Neutron Detector)
F4:n (500<610)
SD4 1
FM4 -1 3 105
\end{lstlisting}
\begin{align}
  \label{eqn:RPM8InteractionRate}
  I_{\text{sim}} &= S_0 I \\
  &= \SI{2.3E3}{neutron\per\second} I
\end{align}

$I_{\text{sim}}$ provides the total number of simulated neutron interactions in the detector.
However, not all of these interactions will lead to counts above the pulse height discriminator setting necessary for meeting the gamma intrisinic efficinecy.
This is corrected for by scaling $I_{\text{sim}}$ by the fraction of counts, $\eta$, that occur above the gamma LLD \eqref{eqn:FractionOfCountsDefination}, \eqref{eqn:RPMCountRate}.
\todo[inline]{An alternative formulation might be to find the count rate per mass \iso[252]{Cf} above the gamma LLD.  This would avoid the low energy noise counts found by only using the lead irridiator spectra, which will dramatically decrease the fraction of counts above the gamma LLD}
\begin{align}
  \label{eqn:FractionOfCountsDefination}
  \eta \equiv \frac{\int_{MLLD}^\infty p(x)dx}{\int_0^\infty p(x)dx}
\end{align}
\nom{$p(x)$}{Measured spectra, as a function of channel number}
\begin{align}
 \label{eqn:RPMCountRate}
 \text{Count Rate} &= I_{\text{sim}} \eta
\end{align}

\pagebreak
\section{Examples}
\begin{Exercise*}[label={PEN LiF with ADS},title={PEN LiF film with ADS},name={}]

Two PEN films of \todo{Composition from Rohit} (\SI{40}{\um} and \SI{150}{\um}) were simulated in the geometry of a simple layered detector design.
This designs consits of alternating layers of film spaced by \SI{1}{\mm} of arylic PMMA spacer\footnote{It has been shown by optimization that this is not a very efficient geometry}.
Confidence in the simulation is provided by also simulating the two films in the neutron irridiator and comparing those values to the observed count rate.
\begin{table}[h]
  \centering
  \caption{PEN with ADS Simulated Interaction Rate and Count Rate}
  \label{tab:PENADPerformace}
  \begin{tabular}{c | c c}
    & Cell Interaction Rate & Count Rate \\
    \hline
    \hline
   \SI{40}{\um}  RPM8 & \num{1.16244E-03} $\pm$ 1.07\% & \\
   \SI{150}{\um} RPM8 & \num{2.55661E-03} $\pm$ 0.0106\% & \\
   \hline
   \SI{40}{\um}  Irridiator & \num{2.88E-5} & 29.6 \\
   \SI{150}{\um} Irridiator & \num{1.05E-4} & 108 \\
  \end{tabular}
\end{table}

by the product of the source strength, interaction rate and cell volume (as in this simulation the tally wasn't normalized by the volume) as shown in Table \ref{tab:PENADSMiller} and \eqref{eqn:MillerCountRate}.
The total number of interactions in the lead well was then subtracted from the cadmium well in order to get the net number of interactions.


\end{Exercise*}

% Bibliography
\bibliography{../Zotero}

\end{document}

