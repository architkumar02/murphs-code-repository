% Intrinsic Efficiency
\documentclass[draftcls,onecolumn]{IEEEtran}

%% INCLUDING THE PREAMBLE
%%%%%%%%%%%%%%%%%%%%%%%%%%%%%%%%%%%%%%%%%%%%%%%%%%%%%%%%%%%%%%%%%%%%%%%%%%%
%                                                                         %
%                                 PREAMBLE                                %
%                                                                         %
%%%%%%%%%%%%%%%%%%%%%%%%%%%%%%%%%%%%%%%%%%%%%%%%%%%%%%%%%%%%%%%%%%%%%%%%%%%

%% PACKAGES
\usepackage[]{lineno}
%\linenumbers
\usepackage[usenames,dvipsnames]{xcolor}
\usepackage{microtype}
\usepackage[obeyDraft]{todonotes}
\usepackage{fancyvrb}
\VerbatimFootnotes
\usepackage{algorithmic}

%% GRAPHICS RELATED
\usepackage{graphicx}
\usepackage[outdir=./tmp/]{epstopdf}
\graphicspath{{../images/}{./}{./tmp/}}
\DeclareGraphicsExtensions{.eps, .pdf, .jpeg, .png,}

%% CPATION SETUP
\usepackage{float}
\usepackage{caption}
\usepackage{subcaption}
\captionsetup{belowskip=12pt,aboveskip=4pt}


%% BIBLIOGRAPHY
\bibliographystyle{ieeetr}

%% UNITS
\usepackage{siunitx}

%% EQUATIONS
\usepackage{amsmath}
%\numberwithin{equation}{section}

%% HYPERLINKS
\usepackage[debug]{hyperref}

%%%%%%%%%%%%%%%%%%%%%%%%%%%%%%%%%%%%%%%%%%%%%%%%%%%%%%%%%%%%%%%%%%%%%%%%%%%
%                                                                         %
%                             Listing Setup                               %
%                                                                         %
%%%%%%%%%%%%%%%%%%%%%%%%%%%%%%%%%%%%%%%%%%%%%%%%%%%%%%%%%%%%%%%%%%%%%%%%%%%
\usepackage{listings}
\lstset{ %
    language=C++,
    basicstyle=\footnotesize\ttfamily,
    numbers=left,
    numberstyle=\tiny\color{gray},
    stepnumber=2,
    numbersep=5pt,
    backgroundcolor=\color{white},
    showspaces=false,
    showstringspaces=false,
    showtabs=false,
    frame=single,
    rulecolor=\color{black},
    tabsize=2,
    breaklines=true,
    breakatwhitespace=false,
    title=\lstname,
    keywordstyle=\color{blue},
    commentstyle=\color{OliveGreen},
    stringstyle=\color{orange}
}
\DeclareCaptionFont{white}{\color{white}}
\DeclareCaptionFormat{listing}{\colorbox[cmyk]{0.43, 0.35, 0.35, 0.01}{\parbox{\dimexpr\textwidth-2\fboxsep\relax}{#1#2#3}}}
\captionsetup[lstlisting]{format=listing,labelfont=white,textfont=white,singlelinecheck=false,margin=0pt,font={bf,footnotesize}}
%\lstnewenvironment{code}[1][]%
%{ \noindent\minipage{\linewidth}
%	\lstset{#1}
%}
%{\endminipage}
%% USER COMMANDS
\usepackage{isotope}
\newcommand{\iso}{\isotope}
\newcommand{\figurewidth}{\textwidth}
\newcommand{\micron}{$\mu$m}


% index generation
\usepackage{makeidx}
\usepackage{exercise}
\makeindex
 
% 'list of notations' generation
\usepackage[refpage]{nomencl}  % refer to the page where notation appears
\newcommand{\definevar}[2]{#1 #2\nomenclature{#1}{#2}}
\renewcommand{\nomname}{List of Notations}
\renewcommand*{\pagedeclaration}[1]{\unskip\dotfill\hyperpage{#1}}
\makenomenclature
 


%% Start of the document
\begin{document}
\title{Simulation of DHS Detector Performance}
\author{Matthew J. Urffer}
\date{\today}
\maketitle

% Nomenclature
\printnomenclature
\printindex

% Tables of Contents, Figures, Tables
\listoftodos
\tableofcontents
\listoffigures
\listoftables
\lstlistoflistings
\section{Introduction}

The performance of films is simulated with in MCNPX, a monte carlo transport code\cite{pelowitz_mcnpx_????}.
The geometry is as in the PNNL reports, namely a nano-gram of \iso[252]{Cf}  encased in \SI{0.5}{\cm} of lead and \SI{2.5}{\cm} of HDPE. 
The size of the RPM8 is \SI{12.7}{\cm} deep, by \SI{30}{\cm} wide and \SI{2}{\m} tall.

An F4 tally (an track length of the cell flux) is modified with a tally muliplier card, an of which is supplied in Listing \ref{lst:InteractionTallyExample}.
A SD card is used to normalize the tally by the volume.
\begin{listing}
  \label{lst:InteractionTallyExample}
\end{listing}
\section{Count Rate Calculation}
The count rate is calculated by computing the total number of itneractions in the detector (multiplying the interaction rate by the source strenth) along with the fraction of counts that are above the gamma pulse height discriminator as found by measurment.
This is shown in equation \eqref{eqn:CountRate} where $S_0$ is the source strength, \nom{$S_0$}{Source strength}, $\Re$ is the MCNPX calculated interaction rate \nom{$\Re$}{Reaction rate}, and $\eta$ is the fraction of counts aboe the gamma LLD \nom{$\eta$}{Fraction of counts above gamma lld}.
\begin{align}
 \label{eqn:CountRate}
 \text{Count Rate} &= S_0 \Re \eta
\end{align}

\pagebreak
\subsection{Examples}
\begin{Exercise*}[label={PEN LiF with ADS},title={PEN LiF film with ADS},name={Example}]

A PEN film was simulated in the geometry of a simple layered detector design.
The validation of the simulation was completed by also simulating the two films in the neutron irridiator and comparing those values to the observed count rate.
The film's count rate was calculated in each well by the product of the source strength, interaction rate and cell volume (as in this simulation the tally wasn't normalzied by the volume) as shown in Table \ref{tab:PENADSMiller} and \eqref{eqn:MillerCountRate}.
The total number of interactions in the lead well was then subtracted from
\begin{table}
  \caption
  \label{tab:PENADPerformace}
  \begin{tabular}{c | c c}\\
    & Interaction Rate & Count Rate
    \hline
    \hline
   \SI{40}{\um}  DHS & 1.16244E-03 0.0107 & \\
   \SI{150}{\um} DHS & 2.55661E-03 0.0106 & \\
   \hline
   \SI{40}{\um} 
                         8.10732E-02
                 3.62054E-04 0.0142
\end{table}

Two differet


The neutron intrinsic efficiency is calculated according to \eqref{eqn:intEffDef} with the source strength as supplied in \eqref{eqn:Cf252SourceStrength}, with the source aging $t = 3.76 \text{yr}$ to the measurement.
\begin{align}
    S &= \SI{1.357E6}{neutron\per\second} e^{-\frac{ \ln{2}}{\SI{2.54}{year}}\SI{3.76}{year}} \\ \notag
      &= \SI{4.86E5}{neutron\per\second}
\end{align}
Approximating the sample as a cylinder (maintaining the area of the faces) yields a radius of \SI{0.95}{\cm}.
\begin{align}
	r &= \sqrt{\frac{\SI{1.3}{\cm} \times \SI{2.2}{\cm}}{\pi}} \\ \notag
	& = \SI{0.95}{\cm}
\end{align}
From Table \ref{tab:NeutronSolidAngle} the closest radius is \SI{1}{\cm}.
Linear interpolation is used to calculate the expected solid angle at a thickness of \SI{0.27}{\cm}, and then a ratio of the areas is used to calculate the solid angle at a radius of \SI{0.95}{\cm}.
\begin{align*}
	\frac{A_1}{A_2} &= \frac{\Omega_1}{\Omega_2} \\ \notag
	\Omega_2 &= \frac{r_2^2 \Omega_2}{r_1^2} \\
	 &= \frac{\SI{0.95}{\cm}^2 \num{6.73E-4}}{\SI{1}{\cm}^2}
	 &= \num{6.07E-4}
\end{align*}
This is then multiplied by the source strength on the date of the measurement, \SI{4.86E5}{neutron\per\second}, to determine the flux.
\begin{align*}
 	N_i &= \Omega S \\
	 &= \num{6.07E-4} \; \SI{4.86E5}{neutron\per\second} \\
   &= \SI{284}{neutron\per\second}
\end{align*}

The \iso[60]{Co} source has aged 459 days (1.26 years) since the measurement.
The source strength is then:
\begin{align}
  S &= \SI{7.178E6}{photon\per\second}\;e^{-\frac{ \ln{2}}{\SI{5.27}{year}}\SI{1.26}{year}} \\ \notag
    &= \SI{6.08E6}{photon\per\second}
\end{align}
Once again interpolation is used to calculated the solid angle.
\begin{align*}
	\frac{A_1}{A_2} &= \frac{\Omega_1}{\Omega_2} \\ \notag
	\Omega_2 &= \frac{r_2^2 \Omega_2}{r_1^2} \\
	 &= \frac{\SI{0.95}{\cm}^2 \num{6.17E-3}}{\SI{1}{\cm}^2}
	 &= \num{5.57E-3}
\end{align*}
This is then multiplied by the source strength on the date of the measurement, \SI{6.08E6}{photon\per\second}, to determine the flux.
\begin{align*}
 	N_i &= \Omega S \\
	 &= \num{5.57E-3} \; \SI{6.08E6}{photon\per\second} \\
      &= \SI{33.9E4}{neutron\per\second}
\end{align*}
It is then possible to plot the intrinsic efficiency in your favorite computational package.

\end{Exercise*}
\begin{Exercise*}[label={LiquidSample},title={Liquid Sample},name={Example}]
The neutron fluence over a \SI{5}{\milli\liter} vial is desired in order to compute the intrisinic efficiency of neutron samples fabricated and measured by members of SCSU.
MCNPX calculations for the solid angle, $\Omega$ show that \SI{4.477E-3}{neuron\per source particle} cross the sample in the lead well, and \SI{1.475E-3}{neuton\per source particle} cross the sample in the cadium well.
Thus the net particles crossing the detector is $\Omega_{\text{net}} =\SI{4.477E-3}{neuron\per source particle} - \SI{1.475E-3}{neuton\per source particle} =\SI{3.002E-3}{neuton\per source particle} $.
Depending on when the sample was measured the number of neutrons crossing the sample can be calculated by multipy \eqref{eqn:Cf252SourceStrength} by the solid angle calculated above.

The SCSU samples where not measured in the \iso[60]{Co} irridiator but rather by placing a source on the outside of the sample.

\end{Exercise*}

% Bibliography
\bibliography{../Zotero}

\section{Appendix}
The generated MCNPX tables was based upon a script input deck for gamma and neutrons in \verb+MCNP/IncidentFlux+.
An analysis script was written in python in order to extract the necessary lines from the output decks.
\end{document}

