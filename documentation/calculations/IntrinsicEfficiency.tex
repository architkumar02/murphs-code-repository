% Intrinsic Efficiency
\documentclass[draftcls,onecolumn]{IEEEtran}

%% INCLUDING THE PREAMBLE
%%%%%%%%%%%%%%%%%%%%%%%%%%%%%%%%%%%%%%%%%%%%%%%%%%%%%%%%%%%%%%%%%%%%%%%%%%%
%                                                                         %
%                                 PREAMBLE                                %
%                                                                         %
%%%%%%%%%%%%%%%%%%%%%%%%%%%%%%%%%%%%%%%%%%%%%%%%%%%%%%%%%%%%%%%%%%%%%%%%%%%

%% PACKAGES
\usepackage[]{lineno}
%\linenumbers
\usepackage[usenames,dvipsnames]{xcolor}
\usepackage{microtype}
\usepackage[obeyDraft]{todonotes}
\usepackage{fancyvrb}
\VerbatimFootnotes
\usepackage{algorithmic}

%% GRAPHICS RELATED
\usepackage{graphicx}
\usepackage[outdir=./tmp/]{epstopdf}
\graphicspath{{../images/}{./}{./tmp/}}
\DeclareGraphicsExtensions{.eps, .pdf, .jpeg, .png,}

%% CPATION SETUP
\usepackage{float}
\usepackage{caption}
\usepackage{subcaption}
\captionsetup{belowskip=12pt,aboveskip=4pt}


%% BIBLIOGRAPHY
\bibliographystyle{ieeetr}

%% UNITS
\usepackage{siunitx}

%% EQUATIONS
\usepackage{amsmath}
%\numberwithin{equation}{section}

%% HYPERLINKS
\usepackage[debug]{hyperref}

%%%%%%%%%%%%%%%%%%%%%%%%%%%%%%%%%%%%%%%%%%%%%%%%%%%%%%%%%%%%%%%%%%%%%%%%%%%
%                                                                         %
%                             Listing Setup                               %
%                                                                         %
%%%%%%%%%%%%%%%%%%%%%%%%%%%%%%%%%%%%%%%%%%%%%%%%%%%%%%%%%%%%%%%%%%%%%%%%%%%
\usepackage{listings}
\lstset{ %
    language=C++,
    basicstyle=\footnotesize\ttfamily,
    numbers=left,
    numberstyle=\tiny\color{gray},
    stepnumber=2,
    numbersep=5pt,
    backgroundcolor=\color{white},
    showspaces=false,
    showstringspaces=false,
    showtabs=false,
    frame=single,
    rulecolor=\color{black},
    tabsize=2,
    breaklines=true,
    breakatwhitespace=false,
    title=\lstname,
    keywordstyle=\color{blue},
    commentstyle=\color{OliveGreen},
    stringstyle=\color{orange}
}
\DeclareCaptionFont{white}{\color{white}}
\DeclareCaptionFormat{listing}{\colorbox[cmyk]{0.43, 0.35, 0.35, 0.01}{\parbox{\dimexpr\textwidth-2\fboxsep\relax}{#1#2#3}}}
\captionsetup[lstlisting]{format=listing,labelfont=white,textfont=white,singlelinecheck=false,margin=0pt,font={bf,footnotesize}}
%\lstnewenvironment{code}[1][]%
%{ \noindent\minipage{\linewidth}
%	\lstset{#1}
%}
%{\endminipage}
%% USER COMMANDS
\usepackage{isotope}
\newcommand{\iso}{\isotope}
\newcommand{\figurewidth}{\textwidth}
\newcommand{\micron}{$\mu$m}


% index generation
\usepackage{makeidx}
\usepackage{exercise}
\makeindex
 
% 'list of notations' generation
\usepackage[refpage]{nomencl}  % refer to the page where notation appears
\newcommand{\definevar}[2]{#1 #2\nomenclature{#1}{#2}}
\renewcommand{\nomname}{List of Notations}
\renewcommand*{\pagedeclaration}[1]{\unskip\dotfill\hyperpage{#1}}
\makenomenclature
 


%% Start of the document
\begin{document}
\title{Intrinsic Efficiency Calculations}
\author{Matthew J. Urffer}
\date{\today}
\maketitle

% Nomenclature
\printnomenclature
\printindex

% Tables of Contents, Figures, Tables
\listoftodos
\tableofcontents
\listoffigures
\listoftables
\lstlistoflistings

The intrisinis efficiency is defined as \eqref{eqn:intEffDef} \cite{knoll_radiation_2009}.
\begin{align}
  \epsilon_{int} = \frac{N_c}{N_i}
\end{align}
where:
\begin{itemize}
  \item[] \definevar{$\epsilon_{int}$}{intrisinic efficinecy},
  \item[] \definevar{$N_c$}{number of counts recorded by the detector}, and
  \item[] \definevar{$N_i$}{quanta of radiation incident upon the detector}.
\end{itemize}
The quanta of radation incident upon the detector can be subdived into two components: the source strength and the solid angle.
The composition of $N_i$ is shown in \eqref{eqn:QuantaIncidentDef}
\begin{align}
  \label{eqn:QuantaIncidentDef}
  N_i = \Omega S_0
\end{align}
where:
\begin{itemize}
  \item[] \definevar{$S$}{source strength}, and 
  \item[] \definevar{$\Omega$}{solid angle}.
\end{itemize}
In general the source strength according the half-life of the source \eqref{eqn:HalfLife}, where \definevar{$S_0$}{initial source strength}, \definevar{$t_{1/2}$}{half life} and \definevar{$t$}{age of source}.
\begin{align}
  \label{eqn:HalfLife}
  S = S_0 e^{-\frac{\ln{2}}{t_{1/2}} t}
\end{align}

The solid angle factor is computed using MCNPX. 
A F1 tally is used over the detector surface with two cosine bins, $-1<\cos\theta<0$ and $0<\cos\theta<1$.
As macrobodies are used for the surfaces of the detector, $-1<\cos\theta<0$ represents the particles that cross into the surface and $0<\cos\theta<1$ the particles that leave the surface.
The MCNPX simulation was benched marked against GS20 and against polymer films.
Tables are provided for common geometries, other geometries can be found by interpolatation.

\subsection{Neutron Intrinsic Efficiency}
The number of counts upon a detector is measured using the neutron irridiator facility.
The quanta of radiation incident upon the detector is found from MCNPX calculations.
This consits of two parts: 1) determining the number of neutrons crossing the detector surface in the lead and cadmium wells and, 2) determining the sourse strength.

The \iso[252]{Cf} source was \SI{0.59}{\ug} on July 2, 2009.
Given that the half-life of \iso[252]{Cf} is 2.65 years and \iso[252]{Cf} has a spontanous neutron emission rate of \SI{2.3E6}{neutron\per\second} the source strength at any given time can be calucalted as \eqref{eqn:Cf252SourceStrength}.
\begin{align}
  \label{eqn:Cf252SourceStrength}
  S &= S_0 e^{-\frac{\ln{2}}{t_{1/2}} t} \\ \notag 
    &= \SI{0.59}{\ug} \iso[252]{Cf} \frac{\SI{2.3E6}{neutron\per\second}}{\si{\ug} \iso[252]{Cf}} e^{-\frac{ \ln{2}}{\SI{2.54}{year} }}  \\ \notag
    &= \SI{1.357E6}{neutron\per\second} e^{-\frac{ \ln{2}}{\SI{2.54}{year} }} 
\end{align}

The following table sumerizes the incident flux for a number of differnet detector sizes and heights.


It should be noted that there is considerable variation in the neutron flux in the detector wells, as shown in \ref{fig:NeutronFluxProfiles}.
Thus, even though the calculations are accurate to less than a precent, the real error on the intrisinic efficinecy will be much higher due to undertainity in where the detector was placed in the well.
\begin{figure}
  \label{fig:NeutronFluxProfiles}

  \caption{Neutron Flux Profiles of the Lead and Cadimum Wells}
\end{figure}
\subsection{Gamma Intrinsic Efficiency}

The gamma intrinisic efficiency is calculated by a combination of simulation to determine the solid angle that the detector subtends and radioactive decay.
The gamma irridator consits of a \SI{97}{\micro Ci} \iso[60]{Co} on January 1st, 2012.
If the incident flux is calcualted according to \eqref{eqn:Co60SourceStrength}.
\begin{align}
  \label{eqn:Co60SourceStrength}
  S &= S_0 e^{-\frac{\ln{2}}{t_{1/2}} t} \\ \notag 
    &= \SI{97}{\micro Ci} \iso[60]{Co} \frac{\SI{1.3E10}{decay\per\second}}{\si{Ci}} \frac{2\text{photon}}{decay}e^{-\frac{ \ln{2}}{\SI{5.27}{year} }}  \\ \notag
    &= \SI{7.178E6}{photon\per\second} e^{-\frac{ \ln{2}}{\SI{5.27}{year} }} 
\end{align}

The gamma irridiator detector well is encased in a 1/2 inch steel pipe which is surrounded by lead, providing a beam like geometry while also introducing lower energy photons. 
The contribution of these lower energy photons is shown if Figure \ref{fig:PhotonFluxAllEnergies}.
\begin{figure}
  \label{fig:PhotonFluxAllEnergies}
  \caption{Photon Flux into Detector}
\end{figure}
%The gamma intrisinic efficiency was calculated that the sample was a cylinder of radius 1.27 cm, 2 mm thick in which the MCNPX calculations show 0.00909 photons cross the surface (the majority of them actually being from Co60 and not scattered photon - see attached)*. Scaling linear for the ratio of the areas yields 0.00909*(1.3 cm * 2.2 cm)/(pi*1.27cm^2) = 0.00527 photons per incident particle.  The source strength is 97 uCi on Jan 1st, but as it has a 30 year half life the half life correction was neglected, and the number of photons crossing the surface was calculated as 0.00527*97E-6Ci*3.7E10Bq/Ci = 18,867 photons per second.

The incident neutron quanta was calculated for also calculated for a 2 mm
Cheers,

% Bibliography
\bibliography{../Zotero}
\end{document}

